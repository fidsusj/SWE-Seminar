\begin{abstract}
\section*{Abstract}

Testing is an essential and time-consuming part of any software development. This work is a collaboration that highlights various approaches to how systematic tests can be created to make this process more efficient, clearer, easier or more successful. After the introduction for this purpose eight different methods are dealt with in own subchapters. Each of these chapters deals with the investigated approach, describes the problem, discusses the literature search on the topic, and then compares the approaches found in each case via a synthesis matrix. The approaches are put into practice using the example of a movie manager. Subsequently, the topic described in this subchapter is summarized and a statement regarding the usage for a systematic test generation is made. 

The first topic dealt with is \enquote{systematic generation of acceptance tests that are executable with FitNesse}. It is about generating automatic tests with the tool FitNesse. This approach tries to solve the difficulties of communication among the participants. For this purpose several artifacts like the fit tables are created and used. After the investigation, it is concluded that the approaches can work well depending on the project size and experience, but unfortunately it is highly dependent on the latter in particular. 

The next chapter discusses that it should be possible to generate test cases in early phases of development. \enquote{Transition systems} can help to derive test cases from specifications. The approaches examined here offer industry standard solutions, but the capability of generating robustness tests for fault detection is still a weakness.

The third subchapter deals with \enquote{Testing with a timing component}. The difficulty of integrating individual software components into an overall project is described as a problem. The aspect particularly considered here is the temporal component, since in many real-time systems the exact time between two instruction executions can lead to different results. After an analysis, the conclusion is drawn that this is still a rarely used approach with many non-automated working steps. 

In the next chapter, approaches for model-based systems are examined. First and foremost, these include \enquote{classification trees}. The goal is to derive automatic test cases based on the requirements and the input parameters of a model. In conclusion, classification trees are described as well arranged and close to the requirements, but can potentially produce many test cases. Moreover, it is difficult to apply these approaches to systems without a simulated model.

The fifth subchapter focuses on \enquote{model based testing} and examines approaches to make testing with models better and describes the differences between system models and test models. After the analysis the difficulties in using the tools given in literature are worked out. Nevertheless, it is possible to reduce test time and errors by using them. 

The initial problem for the next chapter with \enquote{Testing functional and nonfunctional requirements in User Requirements Notation} is the risk for wrong test cases during manual test creation. User requirements notation should fulfill requirements during test creation and thus prevent incorrect or incomplete tests. At the end, the problem is discussed that this approach is largely unexplored. Provided that one makes the effort to take additional steps in test case generation, one still achieves a better quality.

In \enquote{Testing Non-Functional Requirements with Risk Analysis} the focus is on risk analysis, because it is precisely here that error-prone components can be discovered. Suitable approaches are then sought. Among the findings here are the importance of automated tests and that tests for non-functional requirements should have more priority. 

The last chapter describes the \enquote{Testing Non-Functional Requirements with Aspects} and puts thereby in the comparison with the previous section the focus on the \enquote{aspects}. This is intended to describe system-wide functionalities in order to be able to deal with concerns at system level. Also here the missing availability of tools and suitable research are criticized. However, the approaches considered seem promising and should be further refined.

Thus, this joint work addresses the merits and difficulties of various test case generation techniques. Common problems identified are the lack of availability of tools or research, but in suitable use cases most approaches help in improving or automating the test cases. A final conclusion on this is drawn at the end of this work.

\end{abstract}