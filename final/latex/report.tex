% ------ BEGIN PREAMBEL ------
\documentclass[a4paper,10pt, bibliography=totocnumbered]{scrreprt}

\usepackage[utf8x]{inputenc}
\usepackage[english]{babel}

\usepackage{graphicx}
\usepackage{pdfpages}
%\usepackage{subfig}
%\usepackage{microtype}
\usepackage{tabularx}
%\usepackage{amsmath, textcomp}

% Custom packages
\usepackage[numbers]{natbib}
\usepackage{longtable}
\usepackage{ragged2e}
%\usepackage{tikz}
%\usetikzlibrary{positioning}
%\usepackage{pdflscape}
%\usepackage{rotating}
\usepackage{textgreek}

\usepackage{csquotes}
\usepackage{listings}
\usepackage{booktabs}
\usepackage{tablefootnote}
\usepackage{float}

% Links
\usepackage{hyperref}
\hypersetup{
    colorlinks=true,        % false: boxed links; true: colored links
    linkcolor=black,        % color of internal links
%    citecolor=green,        % color of links to bibliography
    citecolor=black,        % color of links to bibliography
    filecolor=magenta,      % color of file links
    urlcolor=blue           % color of external links
}

% Glossary
\usepackage[nomain,acronym,toc]{glossaries} 
\makeglossaries
\newglossaryentry{Acceptance Test}
{
	name={Acceptance Test},
	description={Test to validate that a software meets its requirements by simulating how end-users typically conduct business using the system. Identifying and designing acceptance tests may be difficult for nonfunctional requirements. To be validated, they must first be analyzed and decomposed to the point where they can be expressed quantitatively}
}

\newglossaryentry{Acceptance Test Driven Development (ATDD)}
{
	name={Acceptance Test Driven Development (ATDD)},
	description={Use of automated acceptance tests with the additional constraint that these tests are written before the application code get implemented}
}

\newglossaryentry{All-Interface Coverage Criterion (AIC)}
{
	name={All-Interface Coverage Criterion (AIC)},
	description={Test criterion for component based systems, which ensures that each interface is tested once}
}

\newglossaryentry{All-Interface-Event Coverage Criterion (AIEC)}
{
	name={All-Interface-Event Coverage Criterion (AIEC)},
	description={Test criterion for component based systems, which combines edge coverage with logical coverage}
}

\newglossaryentry{All-Interface-Transition Coverage Criterion (AITC)}
{
	name={All-Interface-Transition Coverage Criterion (AITC)},
	description={Test criterion for component based systems, which ensures that each interface between components is tested once and each internal state transition path in a component is toured at least once}
}

\newglossaryentry{Anti-Lock Braking System (ABS)}
{
	name={Anti-Lock Braking System (ABS)},
	description={Detects the tendency of one or more wheels to lock at an early stage during braking and then immediately ensures that the brake pressure is kept constant or reduced. In this way, the wheels do not lock and the vehicle follows the steering. This allows a car to be braked or brought to a standstill safely and quickly}
}

\newglossaryentry{Aspect}
{
	name={Aspect},
	description={System-wide functionality or issues isolated from the main business logic of the program}
}

\newglossaryentry{Aspect Oriented Programming (AOP)}
{
	name={Aspect Oriented Programming (AOP)},
	description={Programming paradigm using aspects to encapsulate crosscutting concerns. New functionality can be added to existing code without modifying it. Therefore, modularity, maintainability and reusability can be improved}
}

\newglossaryentry{AspectC++	}
{
	name={AspectC++},
	description={Aspect-oriented programming extension for the C and C++ programming languages}
}

\newglossaryentry{AspectJ}
{
	name={AspectJ},
	description={Aspect-oriented programming extension for the Java programming language}
}

\newglossaryentry{Classification Tree (CT)}
{
	name={Classification Tree (CT)},
	description={According to CT-method, the input domain of a test object is analyzed on the basis of its functional specification with respect to various aspects regarded as relevant for the test. For each aspect, disjoint and complete classifications are formed. Classes resulting from these classifications may be further classified iteratively. The stepwise partition of the input domain by means of classifications is represented graphically as a tree. Subsequently, test scenarios are formed by combining classes of different classifications}
}

\newglossaryentry{Component Interface Interaction Graph (CIIG)}
{
	name={Component Interface Interaction Graph (CIIG)},
	description={A special CREMTEG, which represents the time-dependent connectivity relationships between these components as well as time-dependent relations inside a component and a component interface}
}

\newglossaryentry{Component State-Based Event-Driven Interaction Behavior Graph (CSIEDBG)}
{
	name={Component State-Based Event-Driven Interaction Behavior Graph (CSIEDBG)},
	description={A special \\CREMTEG, which, beside CIIG and CSIBG, deals with concurrency. Therefore, the CSIBG model is extended as follows: First, each provider service in a component has one event capture to process the incoming external message. Second, each state contains one event handler and each event capture generates one event in one state. Last, all events are handled by one event handler}
}

\newglossaryentry{Component State-Based Interaction Behavior Graph (CSIBG)}
{
	name={Component State-Based Interaction Behavior Graph (CSIBG)},
	description={A special CREMTEG, which additionally to a CIIG reflects the internal behavior of a component by state transitions and internal message edges with time stamps}
}

\newglossaryentry{Component-Based Real-Time Embedded Model-Based Test Graph (CREMTEG)}
{
	name={Component-Based Real-Time Embedded Model-Based Test Graph (CREMTEG)},
	description={The \\CREMTEG models are derived from component level sequence diagrams and component state diagrams involved in the component interactions. Test criteria for generation integration tests are developed from the CREMTEG models}
}

\newglossaryentry{Contract}
{
	name={Contract},
	description={A requirement-level logical expression used to express preconditions and postconditions of an operation}
}

\newglossaryentry{Correlated Active Clause Coverage (CACC)}
{
	name={Correlated Active Clause Coverage (CACC)},
	description={A special form of predicate logic. For each predicate $p$ and each major clause $c_i$  in $C_p$ (set of all clauses in $p$), choose minor clauses , $c_j$  $j$ $\neq$ $i$, so that $c_i$ determines $p$. $c_i$ has two requirements: $c_i$ evaluates to true and $c_i$ evaluates to false. The values chosen for the minor clauses $c_j$ must cause $p$ to be true for one value of the major clause $c_i$ and false for the other, that is, it is required that p($c_i$ = true) $\neq$ p($c_i$ = false)}
}

\newglossaryentry{Coverage Criterion}
{
	name={Coverage Criterion},
	description={A criterion used to limit the amount of test objectives to be generated. It specifies which paths to traverse in the context of the traversal of a transition system}
}

\newglossaryentry{Crosscutting Concerns}
{
	name={Crosscutting Concerns},
	description={Part of a software that affects other parts, impeding the separation of concerns. This can lead to concern scattering (duplications) or tangling (dependencies)}
}

\newglossaryentry{Domain Specific Language (DSL)}
{
	name={Domain Specific Language (DSL)},
	description={Small language, focused on a particular aspect of a software system}
}

\newglossaryentry{Event Sequence Graph (ESG)}
{
	name={Event Sequence Graph (ESG)},
	description={Used to represent system behavior as well as user-system interaction by events. An ESG is a more abstract representation compared to a state transition diagram of a finite-state automaton}
}

\newglossaryentry{Extendend Automation Method (EXAM)}
{
	name={Extendend Automation Method (EXAM)},
	description={Test method used by the AUDI AG and within the Volkswagen AG to perform tests at component and system levels. EXAM defines the process, the roles, and the tools used to model test cases graphically in UML. Test automation in the scope of EXAM comprises the automated generation of platform dependent code and the automated execution of the derived test suite without human interactions}
}

\newglossaryentry{Functional Requirements (FRs)}
{
	name={Functional Requirements (FRs)},
	description={Describe the functions that the software is to execute, e.g. formatting some text or modulating a signal. They are sometimes known as capabilities or features. A FR can also be described as one for which a finite set of test steps can be written to validate its behavior}
}

\newglossaryentry{Goal-Oriented Requirements Language (GRL)}
{
	name={Goal-Oriented Requirements Language (GRL)},
	description={A visual modeling notation for intentions, business goals, and NFRs of many stakeholders, for alternatives that have to be considered, for decisions that were made, and for rationales that helped make these decisions}
}

\newglossaryentry{Implementation Under Test (IUT)}
{
	name={Implementation Under Test (IUT)},
	description={The part of a real system which is to be tested, which should be an implementation of applications, services or protocols}
}

\newglossaryentry{Interaction Overview Diagram (IOD)}
{
	name={Interaction Overview Diagram (IOD)},
	description={A special form of activity diagram used to show control flow. Each node in the IOD represents either an interaction diagram (sequence diagrams) or interaction occurrences that show an operation invocation}
}

\newglossaryentry{JUnit Testing Framework}
{
	name={JUnit Testing Framework},
	description={A simple framework to write repeatable tests. It is an instance of the xUnit architecture for unit testing frameworks}
}

\newglossaryentry{Model Based Black-Box Testing (MBBBT)}
{
	name={Model Based Black-Box Testing (MBBBT)},
	description={Approach to define test scenarios for software developed in a model-based way from two different perspectives and to create consistency between both requirement-based test design and model-based test design}
}

\newglossaryentry{Model-Based Testing (MBT)}
{
	name={Model-Based Testing (MBT)},
	description={A software testing technique that has gained much interest in recent years by providing the degree of automation needed for shortening the time required for testing}
}

\newglossaryentry{Non-Functional Requirements (NFRs)}
{
	name={Non-Functional Requirements (NFRs)},
	description={Act to constrain the solution, sometimes known as constraints or quality requirements. They can be further classified according to whether they are performance requirements, maintainability requirements, safety requirements, reliability requirements, security requirements, interoperability requirements or one of many other types of software requirements}
}

\newglossaryentry{Object Constraint Language (OCL)}
{
	name={Object Constraint Language (OCL)},
	description={A formal language used to describe expressions on UML models. These expressions typically specify invariant conditions that must hold for the system being modeled or queries over objects described in a model. Therefore, OCL is therefore often used to define contracts}
}

\newglossaryentry{Operation}
{
	name={Operation},
	description={An abstract term used to describe a state transition in a transition system. An operation is equal to a use case or to an interaction diagram/interaction occurrence for instance}
}

\newglossaryentry{Separation Of Concerns}
{
	name={Separation Of Concerns},
	description={Programming principle aiming to split the main task of a program into multiple sub tasks and solving them individually}
}

\newglossaryentry{Stakeholders}
{
	name={Stakeholders},
	description={In a software development process, the software systems are built, tested, maintained, enhanced, and paid. All these activities involve a number of people in building the software. Each of these activities has a different group of users working on it, which may have different interests, requirements for making the software. All these different groups of people comprise stakeholders. Therefore, we can define a stakeholder as an architect of an organization, team, or group having an interest in making a product}
}

\newglossaryentry{System Model (SM)}
{
	name={System Model (SM)},
	description={An abstract model primarily created for system development and then used for testing as well}
}

\newglossaryentry{System Under Test (SUT)}
{
	name={System Under Test (SUT)},
	description={A system that is being tested for correct operation}
}

\newglossaryentry{Systems Modeling Language (SysML)}
{
	name={Systems Modeling Language (SysML)},
	description={A general-purpose architecture modeling language for Systems Engineering applications. It supports the specification, analysis, design, verification and validation of a broad range of systems and systems-of-systems. SysML is a dialect of UML 2, and is defined as a UML 2 profile}
}

\newglossaryentry{Test Case}
{
	name={Test Case},
	description={A test case is a documented set of preconditions (prerequisites), procedures (inputs/actions), and postconditions (expected results) which a tester uses to determine whether a SUT satisfies use case requirements or works correctly. A test case can have one or multiple test scripts and a collection of test cases is called a test suite}
}

\newglossaryentry{Test Model (TM)}
{
	name={Test Model (TM)},
	description={An abstract model solely developed for testing}
}

\newglossaryentry{Test Objective}
{
	name={Test Objective},
	description={Purpose of a test. Stating the objectives of testing in precise, quantitative terms supports measurement and control of the test process. A test objective is used in \cite{ClementineNebut2006} as a synonym for test path as a combination of abstract operations to be composed into one test scenario}
}

\newglossaryentry{Test Scenario}
{
	name={Test Scenario},
	description={A Test Scenario is a statement describing the functionality of the application to be tested. It is used for end to end testing of a feature and is generally derived from the use cases. Test scenarios can serve as the basis for lower-level test case creation. A single test scenario can cover one or more test cases. Therefore a test scenario has a one-to-many relationship with the test cases}
}

\newglossaryentry{Timed Usage Model (TUM)}
{
	name={Timed Usage Model (TUM)},
	description={Markov Chain Usage Models (MCUM) extended by time information, that preserve the semantic of the MCUM and support automated test case generation for embedded systems in test environments as they are established in the automotive industry}
}

\newglossaryentry{Transition System}
{
	name={Transition System},
	description={Used to derive test scenarios by helping to generate test objectives through its traversal. Consists of states and transitions, where states are given by contracts and transitions are given by operations}
}

\newglossaryentry{UC-System}
{
	name={UC-System},
	description={A prototype/interpreter-tool used to build a transition system and to derive test objectives from it}
}

\newglossaryentry{UC-SCSystem}
{
	name={UC-SCSystem},
	description={A prototype-tool using use case scenarios to derive executable test scenarios as JUnit tests}
}

\newglossaryentry{Unified Modeling Language (UML)}
{
	name={Unified Modeling Language (UML)},
	description={A standard visual modeling language intended to be used for modeling business and similar processes as well as the analysis, design, and implementation of software-based systems}
}

\newglossaryentry{User Requirements Notation (URN)}
{
	name={User Requirements Notation (URN)},
	description={A modeling language that aims to support the elicitation, analysis, specification, and validation of requirements. URN is the first international standard to address explicitly, in a graphical way and in one unified language, goals and scenarios, and the links between them. URN models can be used to specify and analyze various types of reactive systems as well as telecommunications standards and business processes. URN allows software and requirements engineers as well as business analysts to discover and specify requirements for a proposed system or process (or evolving ones), and analyze such requirements for correctness and completeness. The URN standard combines two sub-languages: GRL for modeling actors and their intentions, and the UCM notation for describing scenarios and architectures}
}

\newglossaryentry{Use Case}
{
	name={Use Case},
	description={A type of scenario that provides and embodies a context during the elicitation of user requirements. Is used in \cite{ClementineNebut2006} as an abstract, requirement-level term used to describe a main functionality of a system}
}

\newglossaryentry{Use Case Map (UCM)}
{
	name={Use Case Map (UCM)},
	description={The UCM visual scenario notation focuses on the causal flow of behavior optionally superimposed on a structure of components. UCM depict the causal interaction of architectural entities while abstracting from message and data details}
}

\newglossaryentry{Use Case Scenario}
{
	name={Use Case Scenario},
	description={A synonym for sequence diagram in case of \cite{ClementineNebut2006}}
}

\newglossaryentry{XML Metadata Interchange (XMI)}
{
	name={XML Metadata Interchange (XMI)},
	description={XML based metadata interchange format gained from an Interaction Overview Diagram to generate atransition system}
}



% ------ BEGIN EDINGER PACKAGES ------
\usepackage{xcolor}
\usepackage{array}
\usepackage{booktabs}
\newcommand{\tabitem}{~~\llap{\textbullet}~~}
\usepackage{multirow}
\usepackage{pdflscape}
\usepackage[pass]{geometry}

% define colors and configure listing
\colorlet{mygray}{black!30}
\colorlet{mygreen}{green!60!blue}
\colorlet{mymauve}{red!60!blue}
\lstset { %
	language=C++,
	backgroundcolor=\color{gray!10},  
	basicstyle=\ttfamily,
	columns=fullflexible,
	breakatwhitespace=false,      
	breaklines=true,                
	captionpos=t,                    
	commentstyle=\color{mygreen},
	emph={aspect, advice, pointcut, before, JoinPoint}, emphstyle={\color{blue}},
	extendedchars=true,              
	frame=single,                   
	keepspaces=true,             
	keywordstyle=\color{blue},      
	language=c++,                 
	numbers=none,                
	numbersep=5pt,                   
	numberstyle=\tiny\color{blue}, 
	rulecolor=\color{mygray},        
	showspaces=false,               
	showtabs=false,                 
	stepnumber=1,                  
	stringstyle=\color{mymauve},    
	tabsize=3,                      
	title=\lstname,
	numbers=left
}
% ------ END EDINGER PACKAGES ------

% ------ BEGIN HAUSBERGER PACKAGES ------
%\usepackage{csquotes}
%\usepackage{float}
% \usepackage{color}

% \definecolor{dkgreen}{rgb}{0,0.6,0}
% \definecolor{gray}{rgb}{0.5,0.5,0.5}
% \definecolor{mauve}{rgb}{0.58,0,0.82}

%\lstset{
%	frame=tb,
%	language=Java,
%	aboveskip=3mm,
%	belowskip=3mm,
%	showstringspaces=false,
%	columns=flexible,
%	basicstyle={\small\ttfamily},
%	morekeywords={UC, context, pre, @pre, post, and, or, not, implies, forall, exists},
%	numbers=none,
%	numberstyle=\tiny\color{gray},
%	keywordstyle=\color{blue},
%	commentstyle=\color{dkgreen},
%	stringstyle=\color{mauve},
%	breaklines=true,
%	breakatwhitespace=true,
%	tabsize=3
%}
% ------ END HAUSBERGER PACKAGES ------

% ------ BEGIN KOCH PACKAGES ------
% Auskommentiert: Wir nutzen utf8; \usepackage[latin1]{inputenc}
% \usepackage[a4paper]{geometry}
% \geometry{verbose,tmargin=2cm,bmargin=2cm,lmargin=2.5cm,rmargin=2.5cm}
% \setcounter{secnumdepth}{3}
% \setcounter{tocdepth}{3}
% \usepackage{babel}
% \usepackage{array}
% \usepackage{longtable}
% \usepackage{float}
% \usepackage{textcomp}
% \usepackage{graphicx}
% \usepackage[unicode=true,
%  bookmarks=false,
%  breaklinks=false,pdfborder={0 0 1},backref=section,colorlinks=false]
%  {hyperref}
% \hypersetup{pdftitle={Titel},
%  pdfauthor={Autor},
%  pdfpagelabels=true}
 
 % ------ END KOCH PACKAGES ------

% ------ END PREAMBEL ------

%% Title Page
\makeatletter
\renewcommand{\maketitle}{\begin{titlepage}
    \vskip 10\p@
    \hbox{
      \vrule depth 0.99\textheight
        \mbox{\hspace{2em}}
      \vtop{
        \vskip 10\p@
        \hspace{4pt}
        \vskip 50\p@
        \begin{flushleft}
          \Large \@author \par
        \end{flushleft}
        \vskip 50\p@
        \begin{flushleft}
          \huge \bfseries \@title \par
        \end{flushleft}
        \begin{flushleft}
          \Large \bfseries \@subtitle \par
        \end{flushleft}
        \vskip 70\p@
        \begin{flushleft}
          \Large \@publishers \par
        \end{flushleft}
        \vskip 50\p@
        \begin{flushleft}
          \Large \@date \par
        \end{flushleft}
        }}
  \end{titlepage}
}
\makeatother

\author{Author 1, Felix Hausberger, Max Edinger, Andre Meyering, Tobias Koch}
\title{Title }
\subtitle{Technical Report}
\publishers{\textbf{Advisor University of Heidelberg}\\ Prof. Dr. Barbara Paech, Astrid Rohmann}
\date{mm dd, 2021}



% Deutsche Absaetze:
\parindent 0pt
\parskip 12pt

\textwidth145mm
\setlength{\oddsidemargin}{0.7cm}
\setlength{\topmargin}{-0.5cm}
\setlength{\textheight}{22.5cm}

\begin{document}
\maketitle

\begin{abstract}
\section*{Abstract}

Testing is an essential and time-consuming part of any software development. This work is a collaboration that highlights various approaches to how systematic tests can be created to make this process more efficient, clearer, easier or more successful. After the introduction for this purpose eight different methods are dealt with in own subchapters. Each of these chapters deals with the investigated approach, describes the problem, discusses the literature search on the topic, and then compares the approaches found in each case via a synthesis matrix. The approaches are put into practice using the example of a movie manager. Subsequently, the topic described in this subchapter is summarized and a statement regarding the usage for a systematic test generation is made. 

The first topic dealt with is "systematic generation of acceptance tests that are executable with FitNesse". It is about generating automatic tests with the tool FitNesse. This approach tries to solve the difficulties of communication among the participants. For this purpose several artifacts like the fit tables are created and used. After the investigation, it is concluded that the approaches can work well depending on the project size and experience, but unfortunately it is highly dependent on the latter in particular. 

The next chapter discusses that it should be possible to generate test cases in early phases of development. "Transition systems" can help to derive test cases from specifications. The approaches examined here offer industry standard solutions, but the capability of generating robustness tests for fault detection is still a weakness.

The third subchapter deals with "Testing with a timing component". The difficulty of integrating individual software components into an overall project is described as a problem. The aspect particularly considered here is the temporal component, since in many real-time systems the exact time between two instruction executions can lead to different results. After an analysis, the conclusion is drawn that this is still a rarely used approach with many non-automated working steps. 

In the next chapter, approaches for model-based systems are examined. First and foremost, these include "classification trees". The goal is to derive automatic test cases based on the requirements and the input parameters of a model. In conclusion, classification trees are described as well arranged and close to the requirements, but can potentially produce many test cases. Moreover, it is difficult to apply these approaches to systems without a simulated model.

The fifth subchapter focuses on "model based testing" and examines approaches to make testing with models better and describes the differences between system models and test models. After the analysis the difficulties in using the tools given in literature are worked out. Nevertheless, it is possible to reduce test time and errors by using them. 

The initial problem for the next chapter with "Testing functional and nonfunctional requirements in User Requirements Notation" is the risk for wrong test cases during manual test creation. User requirements notation should fulfill requirements during test creation and thus prevent incorrect or incomplete tests. At the end, the problem is discussed that this approach is largely unexplored. Provided that one makes the effort to take additional steps in test case generation, one still achieves a better quality.

In "Testing Non-Functional Requirements with Risk Analysis" the focus is on risk analysis, because it is precisely here that error-prone components can be discovered. Suitable approaches are then sought. Among the findings here are the importance of automated tests and that tests for non-functional requirements should have more priority. 

The last chapter describes the "Testing Non-Functional Requirements with Aspects" and puts thereby in the comparison with the previous section the focus on the "aspects". This is intended to describe system-wide functionalities in order to be able to deal with concerns at system level. Also here the missing availability of tools and suitable research are criticized. However, the approaches considered seem promising and should be further refined.

Thus, this joint work addresses the merits and difficulties of various test case generation techniques. Common problems identified are the lack of availability of tools or research, but in suitable use cases most approaches help in improving or automating the test cases. A final conclusion on this is drawn at the end of this work.

\end{abstract}

\tableofcontents

\chapter{Introduction}
Introduction to all chapters.


\chapter{Testing with natural language processing}\label{sec:topic_1}

%\chapter{Acceptance testing with FitNesse}

% Hickl

\chapter{Systematic Generation of acceptance tests that are executable with FitNesse}
\label{sec:topic_2}

\section{Introduction}\label{sec:topic_2_intro}

This chapter focuses on the creation of acceptance tests that are automatically executable using the tool \textit{FitNesse}.
In this first section of the chapter the reason for using this approach is discussed.
Furthermore, the general features of \textit{FitNesse} as well as the needed artefacts like Fit-tables are explained.
An article that focuses on the topic of creating acceptance tests (see glossary for the term \textit{acceptance test}) that are executable with \textit{FitNesse} was provided by the supervisors of the seminar.
\hyperref[Section]{Section~\ref{sec:literature-search}} documents the literature search used to find a different approach.
The two articles are then described in \autoref{sec:el-attar} and in \autoref{sec:longo}.
Moreover, for a better understanding of the presented approaches these chapters include the execution of them on the \textit{MovieManager}.
The following \autoref{sec:comparison} includes the comparison of the two approaches using a synthesis matrix.
\hyperref[Section]{Section~\ref{sec:topic_2_conclusion}} provides a Conclusion including the most important insights of the process and an assessment in which situations the approaches might be suitable.
For a definition regarding the terms acceptance test, acceptance test driven development (ATDD), stakeholders, test cases, test scenario, UML and use case please see the glossary. 

During the software engineering process communication between the developers and the customers is a crucial factor for the success of the product.
A problem for the communication is the different use of documents by the two main stakeholders:
Customers describe their requirements in natural language whereas the developers create code.
Natural language can often be interpreted in different ways, which can lead to unwanted results.
And whereas code is more precise, it is often too technical for the customer.
Therefore, artefacts are needed that are more precise than natural language and can easily be transformed into code.

One such artefact is a \textit{Fit-table}. 
These tables store easily readable information about acceptance test cases and can be fully automatically executed using the testing tool \textit{FitNesse} \cite{fitnesse}.
Creating \textit{Fit-tables} before the development of the software can help the developers to understand the requirements of the customer by implementing the necessary functionality to pass the acceptance tests.
Thus the customers receive a software that satisfies all their mentioned requirements.
\textit{FitNesse} supports the creation and maintenance of \textit{Fit-tables} as well as the automated execution of the tests represented by the tables.
To make this possible \textit{Fixture-Classes} are needed.
These classes connect the input values from the \textit{Fit-tables} to the \textit{System-under-test} and are executed by \textit{FitNesse}.
An overview of the data exchange during the process is shown in  \autoref{fig:overview-fitnesse} on the next page.
The specific steps are explained in the following:

After the user chooses to execute a set of Fit-tables in \textit{FitNesse}, \textit{FitNesse} executes the Fixture-Classes that belong to the selected tables.
These tables can contain two types of values: Input values and expected output values.
The \textit{Fixture-Class} creates an instance of the \textit{System-under-test} and then transfers the input values into it.
Then it extracts the resulting output values from the \textit{System-under-test} and returns them to \textit{FitNesse}.
These extracted output values are then automatically compared by \textit{FitNesse} to the expected output values from the Fit-tables.
If they are the same, the test was successful and the entry of the table receives the colour \textit{Green}.
Otherwise, the affected part of the test failed and the entry receives the colour \textit{Red}.

\begin{figure}[H]
	\centering
	\includegraphics[width=.7\textwidth]{../images/fitnesse-overview.png}
	\caption{Overview of the data exchange in the execution of \textit{Fit-tables} with \textit{FitNesse}. The Fit-tables can be created and maintained in FitNesse.}
	\label{fig:overview-fitnesse}
\end{figure}


\section{Literature search}
\label{sec:literature-search}

The literature search consisted of a search-term-based search as well as forward- and backward-snowballing.
As start article the work of El-Attar and Smith \cite{el-attar} was given by the supervisor of this seminar.
This article presents an approach to create acceptance tests that can be automatically executed using \textit{FitNesse}.
To find more and possibly different approaches the search question was chosen to be:

\begin{center}
\textit{
 Which approaches to systematically generate acceptance tests that are executable with \textit{FitNesse} exist in the literature?
}
\end{center}

\textit{ACM Digital Library} \cite{acm}, \textit{IEEE Xplore} \cite{ieee}, \textit{Springer Link} \cite{springer} and \textit{Science Direct} \cite{elsevier} were used as search platforms as they are the most common platforms in the field of computer science.
The relevance criteria were chosen as follows:
\begin{itemize}
	\item \textit{Criterion 1:} The article describes an approach to systematically generate acceptance tests that are executable with \textit{FitNesse} \textit{or} gives an overview on the use of \textit{FitNesse} in software engineering.
	
	This criterion was chosen to find approaches that are specific to the subject of this chapter.
	Articles that give an overview over the use of \textit{FitNesse} were also accepted because of their potential to classify the found approaches.
	
	\item \textit{Criterion 2:} The article was not published before the year 2009 which is the year that the article by El-Attar and Smith was published.
	
	This criterion was chosen to get a more recent approach than the start article which was by the creation of this chapter already more than 10 years old.
\end{itemize}

\autoref{fig:lit-search-fitnesse} provides an overview for the search-term-based literature search.
As search terms the terms \enquote{acceptance test} and \enquote{FitNesse} were chosen.
These search terms turned out to be specific enough to fit only a manageable amount of articles.
The search resulted in eight relevant articles of which six presented an approach and two gave an overview over the use of \textit{FitNesse}.
Both snowballing searches from the start articles did not result in any more relevant articles that were not already found by the search-term-based search.
The backward-snowballing did not result in any relevant articles due to their publishing date and hence not passing Criterion 2.
One relevant article was found during the forward-snowballing that was already found by the search-term-based search on the platform Springer Link.

\begin{table}[H]
	\caption{Overview of the search-term-based literature search.}
	
	\includegraphics[width=\textwidth]{../images/LitSearchFitnesse.png} 
	
	\label{fig:lit-search-fitnesse}
\end{table}


The articles that gave an overview over the use of FitNesse were not specific about any approaches and only provided general information.
Therefore, none of these articles was used.
As a second approach to compare to the start article, the work by Longo et al. \cite{longo} was chosen.
This article also describes an approach to create acceptance tests that can be automatically executed by \textit{FitNesse}.
The presented approach differs from the approach of the article by El-Attar and Smith in its use of artefacts.
Also it was created by different authors and was published in 2016, so it is a much more recent approach.


\section{Approach 1: Developing comprehensive acceptance tests from Use Cases and robustness diagrams}
\label{sec:el-attar}

\subsection{Description}

El-Attar and Smith \cite{el-attar} introduce an approach to create acceptance tests that can be automatically executed using \textit{FitNesse.}
Their approach is targeted at larger software projects that use a model-based approach such as the use of UML models (see glossary for the term \textit{UML}). 
It was created such that a non-technical person (e.g. a Business Analyst) can execute it during the early phases of the development of the software.
This makes it possible for the developers to follow the approach of Acceptance-Test-Driven-Development (\textit{ATDD}, see glossary) because of the possibility of executing the acceptance tests at any time during the development process.
ATDD helps the developers during the development process to evaluate which requirements are already implemented and  which are yet to be implemented.

The approach starts with use case models and domain models as initial artefacts.
During the whole process of creating the acceptance tests every step is performed completely manually.
The execution of the final acceptance tests is done fully automatically by the tool \textit{FitNesse}.
To help with traceability during the approach the authors created a tool called UCAT.
This tool does not provide any automation support but allows the user to create use case models and \textit{Fit-tables}.
These artefacts can be linked within UCAT which helps to determine which use cases resulted in which acceptance tests.
 \autoref{fig:overview-el-attar} shows the rough structure of the approach.
The exact steps of the approach are described in the following:

\begin{figure}
	\centering
	\includegraphics[width=.7\textwidth]{../images/ElAttarProcess.png}
	\caption{Overview of the steps in the approach of El-Attar and Smith.}
	\label{fig:overview-el-attar}
\end{figure}

In the first step of the approach \textbf{High-Level-Acceptance-Tests (HLATs)} are created.
For this a domain model and a Use Case model with Use case descriptions is required in the approach.
HLATs are more informal than an executable acceptance test which helps the analyst to be as flexible as possible in describing the acceptance tests at this early stage.
Commonly Use Cases contain Use Case descriptions from which the flows of the Use Case can be extracted.
HLATs describe the system's expected behaviour during all of the flows of the use cases from the Use Case model.
Necessary Pre-Conditions and triggers for the flows are also extracted from the Use Case description while the inputs for the flows can be extracted from the domain model.
Expected test results are also denoted in the HLATs.
At this point they do not need to contain specific values and can be written in natural language.
The general structure of a HLAT is shown in Table 1.2.


\begin{table}[H]
	\begin{small}
	\caption{General structure of a HLAT.}
	\renewcommand{\arraystretch}{1.5}
	\begin{tabularx}{\textwidth}{X|X|X}
		\hline
  		\textbf{Test ID} & \textbf{Description} & \textbf{Expected \newline Result} \\
  		\hline
  		Name of the Use Case \& the flow & Preconditions: \newline Inputs: & Expected result in natural language \\
  		\hline
 	\end{tabularx}
 	\renewcommand{\arraystretch}{1}
 	\label{fig:1}
 	\end{small}	
\end{table}


After the creation of the HLATs a robustness analysis is performed.
To achieve this, for every use case a \textbf{robustness diagram} is created.
These diagrams combine the use cases from the use case model with the objects from the domain model.
They contain actors and entities as well as boundary- and control-objects.
For each use case all involved objects and the connections between the objects are displayed.
The involved objects and the communication between them is extracted from the Use Case description.
During the creation of the robustness diagrams necessary \textit{objects} or \textit{attributes of objects} may be identified that are not already part of the domain model.
These should be added to the domain model.
Also missing steps or preconditions in the Use Case description might be found.
If this is the case, the Use Case descriptions should also be updated.
After this step the HLATs should be adapted to fit the updated domain model and Use Case descriptions.

In the last step all the existing artefacts (possibly except the domain models) are used to create the final product of the approach: \textbf{Executable Acceptance Tests (EATs)}.
These acceptance tests are in the form of specific \textit{Fit-tables}.
To achieve this, the HLATs have to be divided into smaller steps using the information about the Usage Scenario from the related use case description.
This step requires human judgment and is not further described.
For each of these steps the control flow in the robustness diagram gets traced.
In this process the objects and attributes of each step's input, preconditions, outputs and postconditions are determined.
These where before stated in natural language in the HLAT and are exchanged in the EAT with more concrete objects and attributes.
The steps combined with the corresponding control flow are manually converted into \textit{Fit-tables}.
\textit{Fit-tables} used in this approach are either \textit{ActionFixtures}, \textit{RowFixtures} or \textit{ColumnFixtures} \cite{Fit-tables}.
These types of \textit{Fit-tables} can be fully automatically executed using the tool \textit{FitNesse}.
The domain models are ideally not required if the steps before where executed properly because the information from the domain models should already be part of the use case descriptions.

Due to the fact that the approach of creating the acceptance tests is done completely manually, the quality of the resulting acceptance tests depends highly on the experience and skills of the person executing the approach.
Therefore, the authors state that an evaluation would be beyond the limitations of their work.
However, they provide a case example by applying the approach to the software \textit{RestoMapper}.
This example is not part of this work because in the following section the execution of the approach is presented with the application \textit{Movie Manager} that is used throughout this whole report.

\subsection{Application}

The approach starts with Use Case and Domain Models.
As those are not already described in this article, they are created for this chapter.
\autoref{fig:use-case-mm} on the next page shows the Use Case Model for the Movie Manager application.
It contains the use cases and shows connections between them.
For example, removing a movie might result in removing a performer if one of the performers that participated in the movie has no movies anymore after the removal.
Such a relation is highlighted in the Use Case Model with the keyword \textit{extend}.
The domain model of the Movie Manager application is shown in \autoref{fig:domain-mm} on the next page.
It contains the entities Movie and Performer as well as the two views that the user can see.

To illustrate the approach the Use Case \textit{Describe a performer} is used.
In the first step of the approach the HLATs for this Use Case need to be created.
The User Scenarios for this Use Case are as mentioned in the User Task table:
\begin{itemize}
	\item Add and describe a performer
	\item Change the description of an existing performer
	\item View performers (possibly sorted)
\end{itemize}

\begin{figure}[H]
	\centering
	\includegraphics[width=.5\textwidth]{../images/ElAttarUseCase.png}
	\caption{Use Case Model for the Movie Manager application.}
	\label{fig:use-case-mm}
\end{figure}



\begin{figure}[H]
	\centering
	\includegraphics[width=.8\textwidth]{../images/ElAttarDomain.png}
	\caption{Domain Model for the Movie Manager application.}
	\label{fig:domain-mm}
\end{figure}



The HLATs for the Use Case \textit{Describe a Performer} are displayed in \autoref{fig:hlats-mm}.
Each HLAT describes the necessary preconditions, inputs and the expected results of one User Scenario.
This information is extracted from Use Case description.

\begin{table}[H]
	\caption{HLATs for the Use Case \textit{Describe a performer} of the Movie Manager application.}	
	\centering
	\includegraphics[width=.9\textwidth]{../images/ElAttarHLATs.png}
	\label{fig:hlats-mm}
\end{table}

In the next step a robustness diagram is created using the information from the Use Case Model and the Domain Model.
The robustness diagram for the Use Case \textit{Describe a movie} is shown in \autoref{fig:robustness-mm}.
A robustness diagram contains the involved objects that communicate with the user.
These are called boundary objects.
An example for a boundary object is \textit{PerformerView} in  \autoref{fig:robustness-mm}.
It also contains control-objects like \textit{RelateToMovie} in  \autoref{fig:robustness-mm} that makes sure that a performer is related to at least one movie.
The last types of objects are the \textit{User} and the entities like \textit{Performer} or \textit{Movie} in \autoref{fig:robustness-mm}.
The resulting robustness diagram is used to find new information for the domain diagram.
For example, CheckMovieRelations needs to find out whether the performer is linked to at least one existing movie.
Therefore, the domain model needs to include a list of related movies for each performer or the number of related movies.
In this example the first variant (a list of related movies) is used in the domain model.
So this specific information does not have to be added.
Overall the robustness analysis does not bring up any new information but possibly could for other examples which is why El-Attar and Smith have included it in their approach.


\begin{figure}[h!]
	\centering
	\includegraphics[width=.8\textwidth]{../images/ElAttarRobustness.png}
	\caption{Robustness Diagram for the Use Case \textit{Describe a performer} of the Movie Manager application.}
	\label{fig:robustness-mm}
\end{figure}



As last step executable acceptance tests are created for each HLAT.
These are created in the form of \textit{Fit-tables}.
For this example so called \textit{ActionFixtures} are chosen but \textit{RowFixtures} and \textit{ColumnFixtures} are also possible for this approach.
\textit{ActionFixtures} contain a Test ID in the first row.
Each of the following rows contains one action like entering a value or pressing a button.
The ActionFixtures for the three HLATs from  \autoref{fig:hlats-mm} are displayed in the \hyperref[labelName]{Tables~\ref{fig:eats1-mm}}, \ref{fig:eats2-mm} and \ref{fig:eats3-mm} on the next pages.
These Fit-tables are the final result of the approach.

\begin{table}[H]
	\caption{Executable Acceptance Tests for the scenario \textit{Describe a performer, new performer} of the Movie Manager application in form of an \textit{ActionFixture}. 
	A placeholder in the form of \textit{...} is used for entering the other possible attributes of a performer to reduce the size of the table.}
	\centering
	\includegraphics[width=.9\textwidth]{../images/ElAttarEATs1.png}
	\label{fig:eats1-mm}
\end{table}

\begin{table}[H]
	\caption{Executable Acceptance Tests for the scenario \textit{Describe a performer, existing performer} of the Movie Manager application in form of an \textit{ActionFixture}.
	A placeholder in the form of \textit{...} is used for entering the other possible attributes of a performer to reduce the size of the table.}
	\centering
	\includegraphics[width=.9\textwidth]{../images/ElAttarEATs2.png}	
	\label{fig:eats2-mm}
\end{table}

\begin{table}[H]
	\caption{Executable Acceptance Tests for the scenario \textit{Describe a performer, view performers} of the Movie Manager application in form of an \textit{ActionFixture}.
	A placeholder in the form of \textit{...} is used for entering the other possible attributes of a performer to reduce the size of the table.}
	\centering
	\includegraphics[width=.9\textwidth]{../images/ElAttarEATs3.png}
	\label{fig:eats3-mm}
\end{table}


\section{Approach 2: A web framework for test automation, user scenarios through user interaction diagrams}
\label{sec:longo}

\subsection{Description}

Longo et al. \cite{longo} create User Scenarios through User Interaction Diagrams (US-UIDs) which then are fully automatically converted into \textit{Fit-tables} that represent the test data for acceptance tests.
To run these acceptance tests a Fixture-Class is needed that connects the test data from the \textit{Fit-table} with the System-under-test.
The US-UIDs are created in a tool provided by the authors.
They contain functional data such as the involved objects, attributes and functions and also explicit User Scenarios provided by the customer.
The User Scenarios provide the test data and combined with the functional data, \textit{Fit-tables} can be automatically created.
The functional data represents the top row of the \textit{Fit-table} and the User Scenarios the specific values.
\autoref{fig:overview-longo} provides an overview over the steps of the approach.
The only step that is executed automatically is marked in this overview.
Each of the steps is explained in more detail in the following.

\begin{figure}[H]
	\centering
	\includegraphics[width=.7\textwidth]{../images/LongoProcess.png}
	\caption{Overview of the steps in the approach of Longo et al.}
	\label{fig:overview-longo}
\end{figure}
\newpage
In the first step the US-UIDs are created.
This step is described in another work by the authors \cite{longo2}.
Each US-UID contains an explicit User Scenario provided by the customer.
The information about the User Scenario is extended by the developer by adding functional information.
For example, the User Scenario could provide a specific value for a variable.
Then the functional information for this value would be the name of the variable.
With this conjunction of explicit and functional values \textit{Fit-tables} can be automatically created.
The first row contains the functional information.
Each row after the first row represents a User Scenario and contains the given values for the functional information (objects, variables, etc.) given in the first row.

Each row can be used as one specific test case.
To execute these test cases a Fixture-Class is needed.
This class has to be written by the developers.
It creates an instance of the System-under-test and uses \textit{Setter methods} to provide the input from the \textit{Fit-table} to the System-under-test.
Through \textit{Getter methods} the resulting values of the System-under-test can be extracted to validate the success of the test.
For the evaluation step the results from the Getter methods are compared to the expected values from the \textit{Fit-table}.
If they are the same, the test was successful.

To evaluate their approach Longo et al. used their tool to automatically create executable acceptance tests from existing US-UIDs.
The developers of the software related to these US-UIDs already manually created test cases for the software.
In the evaluation the authors compared these manually created tests to the tests created by their tool.
To compare both of them the authors used the techniques \textit{code mutation} and \textit{lack of code}.
The technique \textit{code mutation} involved manipulating the values of an array in the software and \textit{lack of code} was executed by removing a class from the software.
By using both techniques failed tests could be found in both test sets.
The second technique also resulted for both test sets in tests that were not executable.
From these results the authors concluded that tests created with their approach can detect test cases that are \textit{successful}, \textit{failed} or \textit{not executable}.

\subsection{Application}

To illustrate the approach of Longo et al. the use case \textit{Describe a performer (new performer)} is used.
In the first step an US-UID has to be created that displays the explicit information of a User Scenario as well as the underlying functional information.
In this type of model the round boxes are states of the system.
The rectangles contain the User's Input whilst the free text in the round boxes describes the system's output.
Arrows are used to assign functional names to the data and to denote transitions between states.
The US-UID for the example is displayed in \autoref{fig:us-uid-mm}.
In the beginning the performerList contains only the performers a, b and c with attributes e, f and g.
After the execution of the US-UID it also contains performer d with attributes x.


\begin{figure}[tbh]
	\centering
	\includegraphics[width=.85\textwidth]{../images/US-UID.png}
	\caption{US-UID for the Use Case \textit{Describe a performer (new performer)} of the Movie Manager application.}
	\label{fig:us-uid-mm}
\end{figure}

In the next step the functional information from the US-UID needs to be connected to the real
objects and attributes of the \textit{System-under-test}. This is done in a Fixture-Class that is marked
with \textit{@Fixture}.
This class has to be written manually.
The data flow during the execution of acceptance tests with FitNesse and the role of the Fixture-Class in this process is described in \autoref{sec:topic_2_intro}.
Inputs are part of the functional data on the start of the arrows in the US-UIDs. 
For the first display of the US-UID PerformerView the Fixture-Class needs methods to move to the next display, to set the starting performer list and to choose
\textit{create Performer}.
Results have to be extracted from the System-under-test using Getter-Methods.
One such result is the updated performer list in the last state of the PerformerView.
This result can be compared to the expected result given in the US-UID.

In the final step a \textit{Fit-Table} is created.
This step is fully automatic with the tool of Longo et al. because it is only remodeling information from the US-UID.
The functional information is placed in the top row whilst the explicit data of the User Scenarios is stored in the following rows.
Each row represents an User Scenario.
The resulting \textit{Fit-table} for the example is displayed in \autoref{fig:fit-longo} on the next page.

\begin{table}[h!]
	\caption{\textit{Fit-table} for a specific User Scenario of the Use Case \textit{Describe a performer (new performer)} of the Movie Manager application. The expected results end with a question mark.}
	\centering
	\includegraphics[width=\textwidth]{../images/LongoFit.png}

	
	\label{fig:fit-longo}
\end{table}

\newpage
\section{Comparison}
\label{sec:comparison}

As in the other chapters the approaches are compared in a synthesis matrix.

\renewcommand{\arraystretch}{1.5}
%	\begin{tabularx}{\textwidth}{>{\hsize=.2\hsize}X X}
%  		\textbf{1a} & Which artefacts and relations between artefacts are used in this approach? Which
%artefacts are created in the course of the approach? How are the artefacts characterized?
% \\
%  		\hline
%  		\textbf{1b} & What is required and/or input for the application of the approach? \\
%  		\hline
%  		\textbf{1c} & What steps does the approach consist of? Which information is used in which step
%and how? What are the results of the individual steps?
% \\
%  		\hline
%  		\textbf{2a} & Which usage scenarios are supported by the approach? \\
%  		\hline
%  		\textbf{2b} & Which stakeholders are supported by the usage scenarios? \\
%  		\hline
%  		\textbf{2c} &  Which knowledge areas from SWEBOK can be assigned to the usage scenarios? \\
%  		\hline
%  		\textbf{3a} & What tool support is provided for the approach? \\
%  		\hline
%  		\textbf{3b} & Which steps of the approach are automated by a tool? Which steps are supported
%by a tool, but still have to be executed manually? Which steps are not supported
%by a tool?
% \\
%  		\hline
%  		\textbf{4a} & How was the approach evaluated? \\
%  		\hline
%  		\textbf{4b} & What are the (main) results of the evaluation? \\
%  		\end{tabularx}
 
\begin{small} 		
\begin{longtable}[h]{p{0.45cm}|p{0.425\textwidth}|p{0.425\textwidth}}
	\caption{Synthesis matrix}
	\label{tab:blub}
	\\    %%%%<===
	\hline
  	\textbf{No.} & \textbf{Approach 1} & \textbf{Approach 2}\\
  	\hline
  		 1a) & Initial artefacts: \textbf{Use Case and Domain Models} are used to create high level acceptance tests and robustness diagrams 
\textbf{Robustness diagrams} combine the information from the use cases and the domain model. During the creation of the robustness diagrams for the use cases objects and attributes may be identified that are missing in the domain model. The domain model is updated with this missing information.
\textbf{High level acceptance tests (HLATs)} deliver an informal description of acceptance tests. They are tables and use keywords that are chosen by their creator. For each flow of a use case from the Use Case model a HLAT is created.
\textbf{Fit-tables} are a form of executable acceptance tests that can be automatically executed by the tool FitNesse. For each HLAT a Fit-table is created using the information about the control flow in the respective robustness diagram.
 & \textbf{User Scenarios through User Interaction Diagrams (US-UIDs)} show the interaction during a User Scenario. They include User-Input, System-Output, states of interaction and transitions between states.
\textbf{Fit-tables} are a form of executable acceptance tests that can be automatically executed by the tool FitNesse. The Fit-tables in this approach need a specific Fixture-Class for each Use Case that allows the flow of information between the Fit-table and the System-under-test.
 \\
 \hline
  1b) & \begin{itemize}
  		 \item Use Case Model \item Domain Model
\end{itemize}  		  & Requirements in a non-specific type \\
	\hline
  1c) & As initial artefacts Use Case models and Domain models are used.
In the first step a HLAT is created for each flow of each use case from the Use Case model. The information about the preconditions, inputs and triggers for the HLATs gets extracted from the domain model.
The second step is the creation of robustness diagrams for the use cases from the Use Case model. These diagrams also include the objects from the domain model and model the communication between those in the specific use case. If objects or attributes are found in this step that are necessary but not yet part of the domain model, they are added to the domain model. All the other models are updated to fit the domain model. In the last step a Fit-table is created for each HLAT using the control flow that can be seen in the robustness diagram. 
Finally the created Fit-tables can be combined with the Use Case from the Use Case model that they belong to.
 & In the first step the US-UIDs are created by using the known requirements.
The customer delivers the User Scenario and the developer add the functional information for the data used in the scenario.
The Fit-tables are created automatically from the information of the US-UIDs. This step is done by the web framework.\\
\hline
  2a) & Business Analysts receive a systematic approach to create acceptance tests in the Fit-syntax.
Customers receive a product that fits their requirements.
Developers receive acceptance tests that they can use to determine which requirements of the customers they have already implemented and which they have to work on.
 & Customers and developers receive an approach to develop acceptance tests together that include User scenarios provided by the customer.
Developers receive acceptance tests that they can use to determine which requirements of the customers they have already implemented and which they have to work on.
 \\
  \hline
  		 2b) & 
  		 	\begin{itemize}
  		 		\item Developer 
  		 		\item Customer 
  		 		\item  Business Analyst
			\end{itemize}  		 
			& \begin{itemize}
  		 		\item Developer 
  		 		\item Customer
				\end{itemize}\\
	 \hline
  		 2c) & Developer: Software Construction (Test-driven development) \newline
Business Analyst \& Customer: Software Testing
 & Customer/Developer: Software Requirements, Software Testing \newline
Developer: Software Construction (Test-Driven Development)
 \\
  		\hline
  		 3a) & The authors provide the tool UCAT in which Use Case Models and Fit-tables can be created and linked. The created Fit-tables can be automatically executed using FitNesse. & A web framework is provided by the authors in which US-UIDs can be created and converted to Fit-tables. These Fit-tables can be executed with FitNesse. \\
  	 \hline
  		 3b) & The tool UCAT serves as an editor to create Use Case models and Fit-tables. Those two artefacts can also be linked in UCAT.
Every other step in the creation of the acceptance tests is done without tool support.
No step of the approach is done automatically.
 & The creation of the US-UIDs is supported by the web framework which serves as an editor.
Converting the US-UIDs to Fit-tables is done fully automatically by the web framework.
Every other step in the creation of the acceptance tests is done without tool support.
 \\
  \hline
  4a) & 
  	\begin{itemize} 
  		\item Case study with an application (RestoMapper)
		\item	No real evaluation 
	\end{itemize}
 & The authors created tests automatically from existing US-UIDs of an existing application using their approach.
The resulting test set was compared to an existing test set that was created manually.
The code of the application was manipulated using code mutation and lack of code.
For the method code mutation the values of an array were changed manually. For lack of code a class was deleted.
 \\
  \hline
  		 4b & The approach can be applied on an example.
Otherwise no evaluation results because the authors state that an evaluation is beyond the limitations of their work.
The reason for this is that the quality of the created tests in this approach highly depends on the experience and skill of the person executing the approach.
 & Code mutation and lack of code resulted for both test sets in failed tests.
Lack of code also resulted in not executable tests.
The authors concluded that the tests created by their approach can successfully classify tests as successful, failed or not executable.
 \\
  \hline
\end{longtable}
\end{small}

Both approaches provide a possible way to create acceptance tests that are executable with the tool \textit{FitNesse}.
El-Attar and Smith utilize use case models and domain models as their initial data while Longo et al. only need a non-specific description of the use cases.
Generally El-Attar and Smith use more artefacts as their approach needs use case models, domain models, high-level acceptance tests and robustness diagrams as intermediate steps to the final representation of the executable acceptance tests.
In the approach of Longo et al. only US-UIDs need to be created which are then automatically converted to \textit{Fit-tables}.
In contrary to El-Attar and Smith the approach of Longo et al. requires the creation of some code in the process:
This is the case because the used \textit{Fit-tables} differ between the two approaches.
While El-Attar and Smith use the specific table types ActionFixture, RowFixture and ColumnFixture, Longo et al. use easier \textit{Fit-tables} that are connected to the System-under-test via a Fixture-Class.
This Fixture-Class has to be written manually.

Both approaches provide a tool to combine artefacts with the resulting acceptance tests which helps traceability.
For both approaches the creation of the executable acceptance tests is still mostly or completely manual.
While the approach of El-Attar and Smith uses no automation during the creation of the executable acceptance tests, the last step of the approach of Longo et al. is fully automatically.
This is possible because the US-UIDs created in the approach of Longo et al. are a different way to display the information of a \textit{Fit-table} and therefore can be directly converted to a \textit{Fit-table}.
The execution of the final acceptance tests is fully automatic for both approaches.

Both approaches involve customer and developers as stakeholders.
The customer delivers the User Scenarios and receives (because of the development of acceptance tests) potentially a final product that is closer to his needs.
The developers receive automatically executable tests that help them during the development process to determine which requirements are already satisfied and which still need to be implemented.
While in the approach of Longo et al. the creation process of the acceptance tests is done by the customer and the developers together, in the approach of El-Attar and Smith a Business Analyst is responsible for this process.

El-Attar and Smith only visualize their approach through an example and state that the approach cannot be validated in their work because it is beyond the limitations of their work.
The reason for this is that all the steps to create the acceptance tests are done manually and therefore depend on the experience and skill of the analyst performing the steps.
Longo et al. include a small evaluation in their work.
They compare the tests created by their approach to tests that are created without guidelines from the same US-UIDs.
During the testing phase they conclude that the tests created by their approach can be classified as \textit{successful}, \textit{failed} or \textit{not executable}.
Also changes in the source code of the System-under-test resulted in failed tests for both of the test sets which leads the authors to the conclusion that both the tests created without guidelines as well as the tests created with their approach can detect fails in the System-under-test.


\section{Conclusion}
\label{sec:topic_2_conclusion}

The literature search showed that creating acceptance tests that are automatically executable with the specific tool \textit{FitNesse} is not a widely researched topic in the literature.
However, approaches exist that differ in their process to create tests.
Two of these approaches were presented in this chapter:

The approach by El-Attar \& Smith is aimed at larger projects and therefore, might not be useful for smaller products.
It requires the creation of a lot of UML models.
If an analyst exists that has experience in creating these models and at least a few of the used models are created anyway in the engineering process, then this approach might be useful.

The second approach by Longo et al. could be used for smaller projects where the customer is heavily involved.
The customers have to be involved because they have to provide the User Scenarios in this approach.
The approach is heavily dependent on the creation of US-UIDs that contain the information of Fit-tables in a different (possibly better) way.
If the developers and customers prefer US-UIDs over Fit-tables and they want to use User Scenarios, this approach might be useful.

Overall, in the considered approaches the creation of acceptance tests is a process that is highly dependent on the experience and skill of the persons involved.
Once the executable tests are created they are an easy way to measure how well the requirements of the customer are implemented.


\chapter{Topic 3}\label{sec:topic_3}

\chapter{Testing with a timing component}\label{sec:topic_4}

\chapter{Testing with a classification tree}\label{sec:topic_5}

% Julian_Groger
\chapter{Topic 6}\label{sec:topic_6}

\chapter{Testing with System Models}
\label{sec:topic_7}

In this chapter, we aim to study the topic \textit{testing with  system models}. Section \ref{sec:Intro7} presents the introduction of this topic. Section \ref{sec:LS} describes how the systematic literature research is carried out to identify the existing approaches for testing with system models. In \autoref{sec:AP1}, the first relevant approach for this topic, which is given from the Chair of Software Engineering of University Heidelberg to be studied, is described and implemented on the example application \textit{Movie Manager}. In \autoref{sec:AP2}, the second relevant approach, which is identified through the literature research, is studied likewise. Then, \autoref{sec:Compar} compares the two relevant approaches selected for this topic. In conclusion, \autoref{sec:Conc} presents the insights gained from this study.

In order to get familiar with \textit{system model (SM), test model (TM), model-based testing (MBT), Unified Modeling Language (UML), Systems Modeling Language (SysML)} and \textit{Object Constraint Language (OCL)}, please visit the glossary.


\section{Introduction}
\label{sec:Intro7}

The key to successful product engineering in the software industry today is in many cases a good quality-assurance and deployment of software systems \cite{Paper1}. In order to ensure the quality of a product and verify that the product meets its requirements, we need software testing. 

Model-based testing (MBT) is a software testing technique that has gained much interest in recent years by providing the degree of automation needed for shortening the time required for testing \cite{Paper1}. It provides automatic generation of tests from models representing the behavior of the system under test (SUT) \cite{matera}. From these models, test cases can be derived directly or via some model transformations following different coverage criteria. 

As our topic is testing with system models, it is important to differentiate the two model types \textit{system model} (SM) and \textit{test model} (TM) with respect to the MBT. In our first approach \cite{Paper1}, which is presented in \autoref{sec:AP1}, the authors explicitly stated that they used SMs instead of TMs. And the difference between SM and TM is explained as follows: \enquote{The difference between the two being that the former is both used for development and testing, whereas the latter is only used for testing} \cite{Paper1}. 

In order to discuss the differences between using SM and TM in particularly, the authors of \cite{Paper1} provide with some additional authors the article \textit{Model-Based Testing using System vs. Test Models – What is the Difference?} \cite{SMvsTM}, which is newer than \cite{Paper1} and presents two model-based testing case study examples which use SMs and TMs, respectively. In \cite{SMvsTM}, the SM case study example is our first approach \cite{Paper1} to be studied within this chapter.

The article \cite{SMvsTM} summarizes the differences between the SMs and TMs as shown in \autoref{fig:smtm}.

\begin{figure} [H] 
\centering
\includegraphics[scale=0.35]{../images/smtm} 
\caption{Sytem models and test models}
\label{fig:smtm}
\end{figure}

From the viewpoint of modeling, one difference between SM and TM is in the way the expected behavior of the SUT is specified with respect to its interfaces; SM provides an internal viewpoint, whereas the TM provides an external viewpoint of the SUT \cite{SMvsTM}. In the terms of reactive systems, TMs provide stimuli and observe the SUT reactions, while the SMs expect the stimuli and provide reactions. Concerning the purpose of modeling, the TMs are developed solely for testing while SMs can be primarily developed for system development (however, SMs are simpler and more abstract than implementation models) and then used for testing as well \cite{SMvsTM}. \\
However, there also some definitions in \cite{SMvsTM}, which conflict with the usage and the way of the models are created in \cite{Paper1}. According to \cite{SMvsTM}, in TM based approaches, implementation is seen as a black-box thus it is hard to give any verdict about how much of the implementation code has been covered by generated test cases unless source code is instrumented for this purpose. Therefore, requirement coverage is mostly used in this case. On the other hand, in SM based approaches, both code and requirement coverage can be observed, again provided that the implementation code is available for such analysis. \\
But as can be seen in \autoref{sec:AP1}, the authors are entirely concerned with the requirement coverage in \cite{Paper1} while using SMs. After stating this, we would like to clarify that for our study we stick to the definition and usage of SM provided in \cite{Paper1}, as it is also suitable with the \autoref{fig:smtm}.

As the SM of the SUT is typically derived from the informal requirements, it is important to trace how different requirements reflect in the models, on different perspectives and on different abstraction levels, and how the generated test cases cover different requirements \cite{Paper1}, \cite{SMvsTM}. For a model-based testing perspective, traceability of requirements means that the generated test cases from the model are linked with the requirements \cite{Paper2}. Thus, traceability of requirements helps to achieve the right level of coverage and shows what requirement has been covered by what test \cite{Paper1}. It also enables us to identify which requirements have been successfully tested and which have resulted in failures \cite{Paper1}. Hereby, traceability of requirements is a pivotal aspect of MBT that allows one to ensure that all requirements have been tested  \cite{matera}.

%\newpage
\section{Literature search}
\label{sec:LS}

This section describes the systematic literature research carried out according to the \textit{Guidelines for Literature Research of the Chair of Software Engineering} \cite{LRGuidelines}. It aims to identify and analyze the scientific articles regarding testing with system models. 

\subsection*{Research question and research strategy}
\label{subsec:RQ}
The main goal of this search is to find relevant approaches for testing with system models, which are similar to the approach provided in the given article \textit{Tracing Requirements in a Model-Based Testing Approach} \cite{Paper1}. 

Based on \cite{Paper1}, which is our start paper for \autoref{sec:topic_7} in this study, we derive the following research question presented in \autoref{tab:RQ}.

\begin{table} [htb] 
\centering
\begin{small}
\caption{Research question and search terms}
\label{tab:RQ}
\setlength{\tabcolsep}{1em}
\begin{tabular}{ l| p{10cm}}
\hline
\textbf{Research Question} & Which approaches exist to generate test cases using system models for the traceability of requirements?\\
\hline
\textbf{Search Terms}  & system model* AND test* AND trac* AND requirement? \\
\hline
\end{tabular}
\end{small}
\end{table}

In order to answer the research question and find relevant articles, we use \enquote{keyword-based search} and \enquote{snowballing} methods. For the keyword-based search method, we determined the search terms as presented in \autoref{tab:RQ}. These search terms were derived from the research question to focus on a certain aspect of the given article \cite{Paper1} and to find a similar approach. 

For the literature research, the digital libraries of the IEEE Xplore (Institute of Electrical and Electronics Engineers)\footnote{IEEE Xplore \url{https://ieeexplore.ieee.org}} and ACM (Association for Computing Machinery)\footnote{ACM Digital Library \url{https://dl.acm.org}} were used as main sources, since they include the most extensive range of examined scientific publications and are widely recognized. 

In this literature research, keyword-based search method and snowballing search method are conducted to find relevant articles in aforesaid libraries. With keyword-based search method, we search for the search terms presented in \autoref{tab:RQ} with different combinations, since each digital library has a different search method, e.g.  advanced search, or command search.

Snowballing search method is applied to the given article \cite{Paper1} to identify further relevant articles. With backward snowballing method, we go through the reference list of \cite{Paper1} and select the articles as relevant, which fulfill the relevance criteria determined during the literature research. The relevance criteria are presented in \autoref{tab:ArticleCriteria}. With forward snowballing method, we go through the articles, which cite the given article, and again select the articles as relevant, if they fulfill the relevance criteria.

%\subsection{Research Strategy}
%\label{subsec:RS}

\begin{table} [htb] 
\centering
\begin{small}
\caption{Criteria for selecting an article as relevant}
\label{tab:ArticleCriteria}
\setlength{\tabcolsep}{1em}
\begin{tabular}{ l| p{12cm}}
\hline
\textbf{No.} & \textbf{Description} \\
\hline
1 & The article is available in English\\
\hline
2  & The article has a clear focus on software engineering\\
\hline
3  & The article was published in the last fifteen years (2005-2020) \\
\hline
4  &  The article contains an approach for the traceability of requirements using system models to generate test cases, i.e. the article answers the research question\\
\hline
\end{tabular}
\end{small}
\end{table}

In order to ensure that the research methodology and its implementation remain consistent and comprehensible, we formulated four relevance criteria as shown in \autoref{tab:ArticleCriteria}. An article is relevant, if it fulfills all criteria. The first two criteria are formulated to check the availability and the subject area of the articles. Since this work is to be written in English, the article to be used should be available in English. To reduce the comprehensive subject area to the relevant area for our topic, the article should have a clear focus on the area of software engineering as described in the second criterion. The third criterion aims to focus on recent work and to limit the search results. The last criterion is formulated to answer the research question and to find relevant articles, which particularly provide similar approaches to the approach presented in \cite{Paper1}.

\subsection*{Research execution and literature results}
\label{subsec:RE}

First, the search was carried out using the keyword-based search method with aforesaid search terms. In \autoref{tab:ResultKeyword}, the results of this search are documented.\\
First column shows the search location, second column shows the date on which the search was carried out, and third column shows the applied restriction of search. Fourth column presents the exact search query of the search terms and thus shows how they were searched in the specified parts of articles. Fifth column shows the number of overall results, and sixth column shows the number of relevant articles. Finally, last column presents the number of articles chosen to be used in this study.

\begin{table} [H] 
\begin{small}
  \begin{center}
  \begin{scriptsize}
\caption{Results of the keyword-based search method}
\label{tab:ResultKeyword}
\begin{tabular}{   m{1cm} | m{1cm} | m{1.6cm} | m{4.8cm} | m{1cm} | m{1.2cm} | m{0.7cm}  }
\hline
\textbf{Search Location} & \textbf{Search Date} & \textbf{Restriction}  & \textbf{Search Terms}  &  \textbf{\# Results}  & \textbf{\# Relevant Articles}  &  \textbf{Used Results} \\
\hline
IEEE & 20.11.20 & 2005-2020, Advanced Search & ((((\enquote{Full Text Only}:\enquote{system model*}) AND \enquote{Document Title}:test*) AND \enquote{Abstract}:trac*) AND \enquote{Full Text Only}:requirement?) & 64 & 2 & 0\\
\hline
ACM & 20.11.20 & 2005-2020, Advanced Search & $[$Full Text: \enquote{system model*}$]$ AND $[$Publication Title: test* $]$ AND $[$Abstract: trac*$]$ AND $[$Full Text: requirement?$]$ AND $[$Publication Date: (01/01/2005 TO 12/31/2020)$]$ & 9 & 0 & 0\\
\hline
ACM & 21.11.20 & 2005-2020, Advanced Search & $[$Abstract: system model*$]$ AND $[$Publication Title: test*$]$ AND  $[$Abstract: trac*$]$ AND $[$Abstract: requirement?$]$ AND $[$Publication Date: (01/01/2005 TO 12/31/2020)$]$ & 46 & 2 & 0 \\
\hline
IEEE & 21.11.20 & 2005-2020, Advanced Search & ((((\enquote{Abstract}:system model*) AND \enquote{Abstract}:requirement?) AND \enquote{Abstract}:trac*) AND \enquote{Document Title}:test*) & 59 & 0 & 0\\
\hline
\end{tabular}
\end{scriptsize}
 \end{center}
\end{small}
\end{table}
\newpage
As shown in \autoref{tab:ResultKeyword}, in all of our searches we searched the search term \enquote{test*}  in the \textit{documentation title} part, and the search term \enquote{trac*} in the \textit{abstract} part of the digital libraries, in order to find the closest approaches as possible to the presented approach in \cite{Paper1}. Although the \enquote{system model} is the most important search term for this topic, it was not written even in the abstract of the given article \cite{Paper1}. Therefore, with considering that, we decided to search this term also in the \textit{full text} part, instead of only in the \textit{abstract} part of the digital libraries. Consequently, the search terms \enquote{system model} and \enquote{requirement} were searched in the first searches only in \textit{full text} part, then in the second searches in \textit{abstract} part of the digital libraries. 

After identifying relevant articles by using keyword-based search method, snowballing method was conducted to the \cite{Paper1}, which is our start paper for this study, in order to find further relevant articles. The results of the search using snowballing method are detailed documented in \autoref{tab:ResultSnowballing}.

\begin{table} [H] 
\begin{small}
\begin{center}
\begin{scriptsize}
\caption{Results of the snowballing search method}
\label{tab:ResultSnowballing}
\begin{tabular}{  m{1.2cm} | m{4.7cm} | m{1.3cm} | m{1.6cm} | m{1.7cm} | m{1.2cm} }
\hline
\textbf{Search Date} & \textbf{Reference} & \textbf{Direction}  & \textbf{Number of Citations}  &  \textbf{\# Relevant Articles}  &  \textbf{Used Results} \\
\hline
\multirow{2}{1.2cm}{19.11.20} &  \multirow{2}{4.7cm}{\cite{Paper1} Tracing Requirements in a Model-Based Testing Approach} & Backward & 15 &  1 &1  \\ \cline{3-6} & & Forward &11 &1 & 0\\ 
\hline
\end{tabular}
\end{scriptsize}
 \end{center}
	\end{small}
\end{table}

Thus, after conducting the keyword-based and snowballing search methods, we could identify six relevant articles presented in \autoref{tab:RelevantArticles}.

\begin{table} [H] 
	\begin{small}
  \begin{center}
  \begin{scriptsize}
\caption{Results of the literature search}
\label{tab:RelevantArticles}
\begin{tabular}{  m{0.5cm} | m{4.7cm} | m{4.6cm} | m{1.2cm} | m{1.2cm}  }
\hline
\textbf{ID} & \textbf{Title} & \textbf{Authors} & \textbf{Year}  & \textbf{Source}  \\
\hline
\cite{Paper1}&Tracing Requirements in a Model-Based Testing Approach & F. Abbors, D. Truscan and J. Lilius, & 2009 & IEEE \\
\hline
\cite{matera}& MATERA - An Integrated Framework for Model-Based Testing& F. Abbors, D. Truscan and A. Baecklund & 2010 & IEEE\\
\hline
\cite{Relevant3} & A Subset of Precise UML for Model-Based Testing &F. Bouquet, C. Grandpierre, B. Legeard, F. Peureux, N. Vacelet, and M. Utting&2007&ACM \\
\hline
\cite{Relevant4}& Requirements Traceability in Automated Test Generation - Application to Smart Card Software Validation & F. Bouquet, E. Jaffuel, B. Legeard, F. Peureux, and M. Utting&2005&ACM \\
\hline
\cite{SMvsTM} & Model-Based Testing using System vs. Test Models – What is the Difference? &  F. Abbors, D. Truscan, J. Lilius, M. Katara, H. Virtanen, A. Jaeaeskelaeinen, and Q. A. Malik& 2010&IEEE\\
\hline
\cite{Paper2} & Requirements Traceability in the Model-Based Testing Process &Eddy Bernard and Bruno Legeard & 2007&DBLP\tablefootnote{dblp computer science bibliography \url{https://dblp.uni-trier.de/}} \\
\hline
\end{tabular}
\end{scriptsize}
 \end{center}
\end{small}
\end{table}
The articles \cite{Paper1}, \cite{matera}, \cite{Relevant3}  and \cite{Relevant4} were identified through the keyword-based search method, and \cite{SMvsTM} and \cite{Paper2} were identified through the snowballing search method. The first identified article \cite{Paper1} through the keyword-based search method was already given us for this study as start paper. Since \cite{matera} and \cite{SMvsTM} are from the same author group of \cite{Paper1}, we didn't choose them for our study as a second approach to examine. 
\newpage
The articles \cite{Relevant3} and \cite{Relevant4} have also common authors, and one of them is the same author of the article \cite{Paper2}. Therefore, we excluded the \cite{Relevant3} and \cite{Relevant4}, and chose only \cite{Paper2}, \textit{Requirements Traceability in the Model-Based Testing Process}, which is closest to the provided approach in the given article \cite{Paper1}, and very suitable for our topic \textit{testing with system models}.

Consequently, \textbf{\textit{Tracing Requirements in a Model-Based Testing Approach}} \cite{Paper1} and \textbf{\textit{Requirements Traceability in the Model-Based Testing Process}}  \cite{Paper2} are the chosen articles to be studied in this \autoref{sec:topic_7}.


\section{Approach 1: Tracing Requirements in a Model-Based Testing Approach}
\label{sec:AP1}

This section describes the approach presented in \textit{Tracing Requirements in a Model-Based Testing Approach} \cite{Paper1}, and its application on the \textit{Movie Manager}, which is an application used by private individuals at home to manage their own film collection \cite{MovieManager}. What this management includes is described as sub-tasks in the user task sheet of Movie Manager in \cite{MovieManager}. 

\subsection{Description}
\label{subsec:DE1}
This approach was presented in 2009 by Fredrik Abbors, Dragos Truscan, and Johan Lilius in \cite{Paper1} for tracing product requirements across a model-based testing process. With this approach, the authors show how the informal requirements of the SUT evolve and are traced to system specifications and from system specification to tests during the test generation process. They also show how the test results are analyzed and traced back the specification of the system. 
%The approach is only described using small examples from a telecommunications case study. In that case, the SUT is a mobile switching server (MSS), which is a network element located in a mobile telecommunication system. The MSS is responsible for keeping track of the location of mobile subscribes (MS) in the network and for connecting calls between MS’s over 2G and 3G networks. The MSS is also responsible for tracking the movement of MS’s during an ongoing call\\
%Figure~\ref{fig:Ap1-1} presents the model-based testing process of this approach.\\

\begin{figure} [h] 
\centering
\includegraphics[scale=0.3]{../images/Ap1-1.png} 
\caption{Overview of the model-based testing process \cite{Paper1} }
\label{fig:Ap1-1}
\end{figure}

As shown in \autoref{fig:Ap1-1}, the model-based testing process starts with the analysis and structuring of the informal requirements (including protocol specifications, standards, user scenarios, etc.) into a \textit{requirements model} via Systems Modeling Language (SysML). Because compared to the pure textual description of requirements, a requirement diagram offers several advantages due to its visual overview, e.g. missing aspects can be easily identified to help identify additional requirements, and the different types of relationships facilitate traceability and promote understanding \cite{netreqdia}. Requirements traceability is built on top of this testing process, since authors want to be able to trace how different parts of the system models relate to the requirements and then to see how different requirements are covered by the generated test cases. Another reason for tracing requirements is that if a requirement changes, it is essential to know how this change is reflected in the models. Therefore, after requirements are structured hierarchically using SysML requirements diagrams, they are traced to different models or parts of the models implementing them. 
 
%Secondly, once the test report becomes available, we would like to be able to identify which requirements have been successfully tested and which have resulted in failures. In addition, for the failed test cases we should be able to trace back from test cases those parts of the SUT specification that generated the failure.
For the modeling, the Unified Modeling Language (UML) is used to specify the SUT. For a successful test derivation, several perspectives of the SUT are modeled; a class diagram is used to specify a \textit{domain model}, a \textit{behavioral model} is used to describe the behavior of the SUT, \textit{data models} are used to describe the message types exchanged between different domain entities, and \textit{domain configuration models} are used to represent specific test configurations using object diagrams. In order to increase the quality of the resulting models, a set of modeling guidelines and validation rules are defined. These rules ensure that the models are consistent with each other and moreover, that they contain the information needed in the later phases of the testing process. For editing the SysML and UML models and for running these validation rules, the NoMagic’s \textit{MagicDraw}\footnote{NoMagic MagicDraw \url{https://www.nomagic.com/products/magicdraw}} tool has been used. 

When all requirements have been linked to model elements and the models have been validated, the models used to specify the SUT are subsequently transformed into input for an automated test derivation tool for model driven testing, \textit{Conformiq Qtronic}\footnote{Conformiq Qtronic \url{https://www.conformiq.com/}}, via an automated transformation. Qtronic accepts as input a SM of the SUT from which it automatically designs test cases according to the selected coverage criteria. The input model can be expressed as a combination of UML state machines \cite{SMvsTM} and the transformation basically translates these UML models to the Qtronic Modeling Language (QML), a textual specification language with a Java-like syntax used by Qtronic for specifying the SUT. There are two main purposes for modeling behavior using state machines. First, by using UML state machines the behavioral properties of SUT specification are formally verified. Second, Qtronic tool expects the behavior of SUT in the form of state machines \cite{traqml}. During the transformation from UML to QML, links between requirements and model elements are preserved. 

In this approach, only the online testing mode of this Qtronic is used, in which tests are generated and applied on-the-fly against the SUT. For test generation, different coverage criteria types can be manually selected from the graphical user interface (GUI) of Qtronic like requirements, state, transitions, paths, conditional, or statement coverage. The generated test cases are sequences of input/output messages and their data values derived from the SMs to be sent/received by the SUT \cite{SMvsTM}. As the designed test cases are at the same abstraction level as the SM, an adapter is used to concretize the tests \cite{SMvsTM}. One-by-one Qtronic generates an input message, sends it via the adapter to the SUT, and generates a new input message based on the responses from the SUT. A logging back-end can be used during test execution. The logging back-end provides connectivity to the Qtronic reporting infrastructure and it is used by Qtronic to generate a test report. Three logging back-ends are provided by default. With these logging back-ends, Qtronic can generate test reports in HTML, SQLite, and XML format. When all tests have been applied against the SUT, Qtronic generates a test report in the chosen format, which summarizes the results of the testing process in terms of generated test cases, verdicts, coverage levels, requirements traceability matrix, etc. Unfortunately, the rest of the automatic test generation and execution process of this approach was not provided by the authors. 

With this approach, it is also possible to trace-back requirements from test cases to models. For this purpose, the rest report is analyzed and the information of the failed test cases is collected. Then, the requirements attached to those test cases are traced back to system models. This enables to identify which requirements were not validated during testing and to what parts of the specification they are linked. A Python script was developed by the authors that automatically analyzes the Qtronic test report. 

According to the authors, the provided approach proved beneficial through the fact that many errors have been detected in the early stages of the process, when the system models have been created. For instance \cite{SMvsTM}, some errors originated from inconsistencies discovered in the SM, which were due to misunderstanding of requirements or to incomplete validation of models before testing. Other errors have been found in the adapter, as well as in the SUT.

Furthermore, it should be noted that although Conformiq is still active and offering different kind of products for an end-to-end test automation, its Qtronic product mentioned in this approach is not existing in Conformiq's current product list. Unfortunately, after our research, we could not find any information about what happened to Qtronic. Therefore, we assume that it was not developed further.





\subsection{Application}

%As presented in Figure~\ref{fig:Ap1-s1}, 
First, functional requirements of the SUT should be analyzed and structured into requirements models using the requirements diagrams of SysML. Each requirement element contains a \textit{name} field which specifies the name of the requirement, an \textit{id} field that simply specifies the id of the requirement, a \textit{text} field which describes the requirement, and a \textit{source} field which specifies the origins of the requirement. The source can be a link to or a name of a textual document from where the requirement has been extracted. 

In our implementation, we converted the user sub-tasks of Movie Manager into system functions in order to build a requirements diagram from system functional requirements. \autoref{fig:Ap1-req} presents the example of a SysML requirements diagram for our application. 


\begin{figure} [h] 
\centering
\includegraphics[scale=0.49]{../images/Ap1-req} 
\caption{Example of a SysML requirements diagram}
\label{fig:Ap1-req}
\end{figure}

%As presented in Figure~\ref{fig:Ap1-s2}, 
Then, the UML models of the SUT are built starting from the requirements models. During this process, the requirements should be traced to different parts of the models to point how each requirement is addressed by the models. In the provided approach, a set of modeling guidelines and validation rules should be defined for ensuring the quality of the resulting models. For this purpose, Object Constraint Language (OCL) is used. The aforementioned NoMagic’s MagicDraw tool has been used for editing the SysML and UML models and for running the validation rules. But in our implementation of the approach, we use the free online tool of the diagram software \textit{draw.io}\footnote{draw.io \url{https://drawio-app.com/}} via \textit{diagrams.net}\footnote{diagrams.net \url{https://www.diagrams.net/}} for this purpose.
\newpage
As the next step, a state machine model should be modeled from the requirements model, which is to be used as input for Qtronic, since the Qtronic tool expects the behavior of SUT in the form of state machines in order to be able to convert it into QML. %In the provided approach, the relationships between requirements and models are specified on several levels. For instance, non-leaf requirements are linked to state machine models, and leaf requirements in the requirements tree are linked to other UML elements to which they apply, e.g. transitions in a state machine or classes in a class diagram. As an exceptional situation, the top-level functional requirements are linked to use cases in the use case model of the SUT. \\
In \autoref{fig:Ap1-sm2}, we present an example for the requirement with the requirement id \enquote{1.4} from the requirements model presented in \autoref{fig:Ap1-req}.

\begin{figure} [H] 
\centering
\includegraphics[scale=0.3]{../images/Ap1-sm2} 
\caption{Example of a UML state machine}
\label{fig:Ap1-sm2}
\end{figure}

But as already mentioned in the description of this approach in ~\autoref{subsec:DE1}, Qtronic is no longer available and therefore we can not apply the remaining steps, which are \textit{automatic test design and execution} and \textit{generation of test report}. As we could not find any information on how Qtronic generates tests, we can not provide any exemplarily test cases either. 




\section{Approach 2: Requirements Traceability in the Model-Based Testing Process}
\label{sec:AP2}

This section describes the approach presented in \textit{Requirements Traceability in the Model-Based Testing Process} \cite{Paper2}, and its application on Movie Manager. 

\subsection{Description}
\label{subsec:DE2}
This approach was presented in 2007 by Eddy Bernard and Bruno Legeard in \cite{Paper2} to automatically produce the traceability matrix from requirements to test cases, as part of the test generation process. Automatically generating the traceability matrix from requirements to test cases implies managing the links between the requirements specification, the model and the test cases. This approach focuses on this problem and is embedded in the \textit{LEIRIOS Test Designer} technology, which is named currently \textit{Smartesting}\footnote{Smartesting \url{https://www.smartesting.com/}} \cite{matera}. The approach tags the dynamic part of the UML model with the requirement identifiers and uses it to produce automatically both the test cases and the traceability matrix. 

First, the validation engineer constructs the model from a textual specification. The goal of this step is to translate the description of the features of the tested system into a precise UML model of the expected behavior \cite{LTG}. The LEIRIOS Test Designer approach uses a subset of the UML 2.0 language with class diagrams, instance diagrams, state machine diagrams, and OCL expressions. While the OCL expressions within the class diagram formally describe the expected behavior of operations of a class using preconditions and postconditions, the OCL expressions within the state machine formalize guards and effects of transitions between the states. 

Then, requirements traceability is managed by tagging manually the postconditions of the operations and the effects of the transitions in the UML model with the requirements. The format uses ad-hoc comment symbols to associate a requirement with an OCL statement which will be involved into a test target during the generation of the test cases. Once the model is reasonably trustworthy, the generation of tests is steered by the validation engineer on the basis of test selection criteria. Test selection criteria are supported by the LEIRIOS Test Designer tool to control the choice of tests from all the possible tests that can be derived from the behavior UML model \cite{LTG}, e.g. transition-based coverage, decision-based coverage, and data-oriented coverage. It is important to notice that all these test selection criteria are related to models and define how well the generated test suite covers the model \cite{LTG}. 

The test generation method is provided by the LEIRIOS Test Designer tool. It consists in testing all the possible behaviors of the specification operations, by traversing the states of the system. This strategy is controlled by the previously defined test selection criteria. 
\newpage
This method is performed as follows:
%
\begin{itemize}[noitemsep]
  \item [--]Partitioning of the model operation to generate all the possible expected behaviors,
  \item [--]Computation of variable domain boundaries from each behavior (called test targets),
   \item [--] Generation of test cases obtained, for each test targets, by traversing the underlying reachability graph of the model from the initial state to reach a state satisfying a test target.
\end{itemize}
Operations are the internal actions that modify the system state during the computation of user actions. They are called from the action part of transitions in the state machine and their meaning is defined by OCL postconditions in the class diagram \cite{LTG}. 

A test case reaches a target, which involves OCL statements tagged with one or several requirement identifiers. Then, LEIRIOS Test Designer makes it possible to match a test with requirements during the generation of the test cases. In LEIRIOS Test Designer, a general framework to convert the generated test cases into executable scripts is also provided. The test script pattern is a source code file in the target language with some tags indicating where sequences of operation invocations have to be inserted. A traceability matrix is obtained after the tests are executed. For each requirement, the matrix gives the list of needed test cases to test it. 

In \cite{Paper2}, this presented approach is implemented on a simplified version of a drink vending machine controller, which is SUT, with \enquote{Buy a drink} scenario to illustrate the test generation process with LEIRIOS Test Designer. After expected functional requirements of the SUT are listed, it is modeled via a \textit{class diagram}, an \textit{instance diagram} and a \textit{state machine}. Expected behaviors are specified by transitions (either external or internal) and those transitions are triggered by a \textit{user event}, a \textit{guard} and an \textit{effect}. Requirements traceability is managed by tagging the effect of each transition with requirement identifiers. This link between the model and the requirements is used by the LEIRIOS Test Designer tool to produce the traceability matrix between generated test cases and requirements. At the end, from the SUT model, 12 test targets are generated, one for each transition, and 12 test cases are generated. 

Unfortunately, the approach lacks the detailed explanation on how the test cases are generated and converted to executable test scripts via LEIRIOS Test Designer tool.




\subsection{Application}

In this part, we aim to to follow the same steps of the approach on our Movie Manager example as similar as possible. 

First, we get the informal requirements of the system under test. To do that, we create use case diagram of Movie Manager using draw.io tool as shown in \autoref{fig:Ap2-2}. Our use case scenario: \enquote{User manages movies and corresponding performer data of a movie collection}.

\begin{figure} [H] 
\centering
\includegraphics[scale=0.35]{../images/Ap2-2} 
\caption{Example for a use case diagram of Movie Manager}
\label{fig:Ap2-2}
\end{figure}

The system under test is the Movie Manager. To precisely define the expected functional requirements of the Movie Manager, a list of requirements including Req Identifier, Req Name, and Req description is defined as presented in \autoref{tab:MMreq}. For the requirements MM-2 and MM-7 we added some preconditions based on the Movie Manager \cite{MovieManager} to be able to apply the next steps of the approach within this section. 

\begin{table} [H] 
  \begin{center}
  \begin{small}
\caption{Movie Manager requirements}
\label{tab:MMreq}
\begin{tabular}{  m{1.3cm} | m{3.8cm} | m{8cm}  }
\hline
\textbf{Req. Id}& \textbf{Req. Name}&\textbf{Req. Description}   \\
\hline
MM-1& Describe\_Movie&The Movie Manager shall allow to describe a movie\\
\hline
MM-2&Remove\_Movie&The Movie Manager shall allow to remove movie. If the movie is linked with a performer, the user is warned by Movie Manager. The movie is removed from the database after user confirms.\\
\hline
MM-3&Describe\_Performer& The Movie Manager shall allow to describe a performer\\
\hline
MM-4&Relate\_Performer\_to\_Movie&The Movie Manager shall allow to relate performer to movie \\
\hline
MM-5&Remove\_Performer& The Movie Manager shall allow to remove performer\\
\hline
MM-6&Manage\_Watched\_Movies& The Movie Manager shall allow to manage watched movies\\
\hline
MM-7& Rate\_Movie\_or\_Performer& The Movie Manager shall allow to rate movie or performer. If the movie or performer does not exist, the user is warned by Movie Manager\\
\hline
\end{tabular}
\end{small}
 \end{center}
\end{table}

Then as an instance, we create the Movie Manager Class Diagram for the aforementioned requirements as presented in \autoref{fig:Ap2-class}.

\begin{figure} [H] 
\centering
\includegraphics[scale=0.42]{../images/Ap2-class} 
\caption{Example for a class diagram of Movie Manager}
\label{fig:Ap2-class}
\end{figure}

The Movie Manager Class Diagram defines the data and the points of control and observation of the application under test. The enumeration \enquote{Messages} represents the possible message/warning to be displayed on the dialog window of the application. For our case, we considered only the messages like, \textit{movie is linked with performer} and \textit{performer does not exist}.

In addition to class diagram, an instance diagram and a state machine are also provided in the use case scenario of the provided approach. In the approach, the test generation method of LEIRIOS Test Designer consists in testing all the possible behaviors of the specification operations, by traversing the states of the system. But in our implementation, we have to skip this step since it is performed by the LEIRIOS Test Designer tool, which is called now Smartesting as already explained in ~\autoref{subsec:DE2}. We cannot use the tool right now, because it requires a scheduling even for the demo version. 

Therefore, we give only exemplarily two different internal activities specified according to the description in the article \cite{Paper2}, one for the user option \enquote{remove movie}, and other one for the user option \enquote{rate performer}, as shown in \autoref{tab:intact}. 

\begin{table} [H] 
	\begin{small}
  \begin{center}
  \begin{scriptsize}
\caption{Example of internal transitions}
\label{tab:intact}
\begin{tabular}{  m{3.3cm} | m{3.4cm} | m{6.4cm}  }
\hline
\textbf{Trigger/Label}& \textbf{Guard}&\textbf{Effect}   \\
\hline
removeMovie (movie)
\newline- Case movie is linked with a performer & Movie-$>$linked (m: movie $|$ m.linked=True) & MESSAGES\textbf{::} MovieIsLinkedWithPerformer \textbf{/*@REQ: MM-2@*/}\\
\hline
ratePerformer (performer)
\newline- Case performer does not exist & Performer-$>$exists (p: performer $|$ p.exists=False) & MESSAGES\textbf{::} PerformerDoesNotExist \textbf{/*@REQ: MM-7@*/ }\\
\hline
\end{tabular}
\end{scriptsize}
 \end{center}
\end{small}
\end{table}

These are the transitions, which are triggered by a user event, a guard and an effect. Requirements traceability is managed by tagging the effect of each transition with requirement identifiers. For example, the requirement MM-7 is used to annotate the effect in case of not existing performer. This is the link between the model and the requirement. \\
As last step, this link between the model and the requirements should be used by the LEIRIOS Test Designer tool to produce the traceability matrix between generated test cases and requirements. For each transition a test target and a test case should be via LEIRIOS Test Designer tool generated. Since we cannot use the tool, we generate the test cases as an example for both transitions manually. \autoref{tab:T1} and \autoref{tab:T2} presents these generated test cases for the two internal transitions showed in \autoref{tab:intact}.

\begin{table} [H] 
  \begin{center}
  \begin{small}
\caption{Test 1: Movie is linked with performer - Covered Requirements: MM-2}
\label{tab:T1}
\begin{tabular}{ m{0.8cm} | m{5cm} | m{7.3cm} }
\hline
\textbf{Step}& \textbf{Operation}&\textbf{Attributes Values}   \\
\hline
1 & MM2.removeMovie (movie)& MM2.display = MovieIsLinkedWithPerformer\\
\hline
\end{tabular}
\end{small}
 \end{center}
\end{table}
\begin{table} [H] 
\begin{center}
 \begin{small}
\caption{Test 2: Performer does not exist - Covered Requirements: MM-7}
\label{tab:T2}
\begin{tabular}{  m{0.8cm} | m{5cm} | m{7.3cm}  }
\hline
\textbf{Step}& \textbf{Operation}&\textbf{Attributes Values}   \\
\hline
1 & MM7.removePerformer (performer) & MM7.display = PerformerDoesNotExist\\
\hline
\end{tabular}
\end{small}
 \end{center}
\end{table}



\section{Comparison}
\label{sec:Compar}

In this section, we describe the special features, similarities and differences of each approaches.

Both approaches are very similar in terms of the way they use the information and of their steps. They both use informal requirements as input for their approaches, use system models for this purpose (UML models including OCL), and make use of requirements traceability for ensuring the quality of the generated test cases. They both offer tool support to generate tests automatically. \\
However, there are also differences between them. In the first approach, \cite{Paper1}, the informal requirements are analyzed and structured into requirements models and then OCL is used for verifying the quality of these requirements models.
In the second approach, \cite{Paper2}, the expected functional requirements of the system under test are derived from a use case scenario. The requirements traceability is managed by tagging manually the postconditions of the operations and the effects of the transitions in the UML model with the requirements via ad-hoc comment symbols in order to associate a requirement with an OCL statement which will be involved into a test target during the generation of the test cases. \\
Furthermore, the LEIRIOS Test Designer tool provided in the second approach does not offer a tool support for tracing-back requirements from tests to models, while the first approach supports it via the automatically generated test report by their provided tool, Qtronic.

\autoref{tab:TSM} presents the detailed comparison of both approaches as a synthesis matrix. In the first column, the number of the synthesis questions are presented. Number 1 implies the description, number 2 implies the benefit, number 3 implies the tools, and number 4 implies the quality for the compared approaches. Then, the second column presents the answers to these questions for approach 1, and third column for approach 2. 

\newpage
\newgeometry{margin=1cm}
\begin{landscape} 
%\begin{scriptsize}
\begin{small}
\begin{longtable}{ p{0.5cm} | p{11cm} | p{11cm} }
\caption{Synthesis matrix}
\label{tab:TSM}
\\    %%%%<===
\hline
\textbf{No.} & \textbf{Approach 1: Tracing Requirements in a Model-Based Testing Approach}  & \textbf{Approach 2: Requirements Traceability in the Model- Based Testing Process} \\
\hline
1a) & - Informal requirements
\newline - Requirements models via requirements diagram of SysML
\newline - UML Models, e.g. class Diagram specifies a domain model, a behavioral model describes the behavior of the SUT using state machines, data models describe the message types exchanged between different domain entities, domain configuration models represents specific test configurations using object diagrams.
\newline - Generated tests and test report
 & - Informal requirements
\newline - UML Models (including OCL) e.g. Use case diagram, class diagram, enumeration diagram, instance diagram, state machine.
\newline - Links between the requirements and models
\newline - Test cases and test scripts
\newline - Traceability Matrix \\
\hline
1b) & Informal requirements are input for this approach.
\newline A set of modeling guidelines and validation rules should also be defined.
\newline In order to automatically transform UML models to the QML via the Qtronic tool, all requirements should have been linked to model elements and the models should have been validated.
\newline The desired coverage criteria used for test generation should be manually selected from the GUI of Qtronic.
\newline The presented approach is for functional requirements and it only supports online testing. & Informal requirements are input for this approach. 
\newline Uncontrollable or unobservable elements should not be modeled.\\
\hline
1c) & 1. The informal requirements are analyzed and structured into a requirements model using the requirements diagrams of the SysML.
\newline 2. SUT is specified using UML (including OCL) and several perspectives of the SUT are modeled, e.g. class diagram, behavioral model, data models, domain configuration models etc. 
\newline 3. A set of modeling guidelines and validation rules are defined using OCL. MagicDraw tool has been used for editing the SysML and UML models and for running the validation rules
\newline 4. The models used to specify the SUT are subsequently transformed into input for the automated test derivation tool Qtronic. 
\newline 5. At the end of each test run, an automatically generated test report will summarize the result of the testing process in terms of generated test cases, verdicts, coverage levels etc.
& 1. UML-Modelling of functional requirements of the application under test, e.g. class diagrams, instance diagrams, state machine diagrams  etc.
\newline2. Tagging manually the post-conditions of the operations and the effects of the transitions in the UML model with the requirements. The format uses ad-hoc comment symbols in order to associate a requirement with an OCL statement which will be involved into a test target during the generation of the test cases.
\newline3. Driving the test generation process in which the generation of tests is steered by the validation engineer on the basis of test selection criteria.
\newline4. Test generation method provided by the LEIRIOS Test Designer tool
\newline5. Executable test script generation which are converted from the generated test cases via LEIRIOS Test Designer. A test script pattern and a mapping table are to be defined by the test engineer.\\
\hline
2a) & Test generation, and tracing requirements to models, to test specification, and back to models again.& Test generation, and tracing requirements to test cases. \\
\hline
2b) & Product engineer, software engineer, requirement engineer and user of the provided tooll/tester.  & Validation engineer, software engineer, requirement engineer and user of the provided tool/tester. \\
\hline
2c) & Chapter 1: Software Requirements
\newline - Requirements Process
\newline - Requirements Analysis
\newline - Practical Considerations (e.g. Requirements Tracing)
\newline Chapter 4: Software Testing
\newline - Test Techniques (e.g. Model-Bases Testing Techniques)
\newline -  Software Testing Tools (e.g. Testing Tool Support and Categories of Tools)
\newline Chapter 9: Software Engineering Models and Methods 
\newline - Modeling
\newline - Types of Models
\newline - Analysis of Models (e.g. Traceability) & Chapter 1: Software Requirements 
\newline - Requirements Process (e.g. Process Actors)
\newline - Requirements Analysis
\newline - Practical Considerations (e.g. Requirements Tracing)
\newline Chapter 4: Software Testing
\newline - Test Techniques (e.g. Model-Bases Testing Techniques)
\newline -  Software Testing Tools (e.g. Testing Tool Support and Categories of Tools)
\newline Chapter 9: Software Engineering Models and Methods 
\newline - Modeling
\newline - Types of Models
\newline - Analysis of Models (e.g. Traceability) \\  
\hline
3a) & The NoMagic’s MagicDraw has been used for editing the SysML and UML models and for running the validation rules. \newline Conformiq’s Qtronic tool is provided for the automated test derivation and test report generation. Only online testing mode of Qtronic is used to generate tests. & LEIRIOS Test Designer tool is provided to generate test cases and to produce the traceability matrix between generated test cases and requirements. The tool is used in the test and test script generation processes.\\
\hline
3b) & The creation of the models is not automated and the desired coverage criteria used for test generation should be manually selected from the GUI of Qtronic. Tests and test reports are automatically generated by the provided tool, Qtronic. &Requirements traceability is managed by tagging manually the postconditions of the operations and the effects of the transitions in the UML model with the requirements via ad-hoc comment symbols. Tests are automatically generated by the provided tool, LEIRIOS.\\
\hline
4a) & Approach does not present an evaluation. Only validation rules have been defined and implemented for both requirements models and for SMs for checking different quality metrics of the resulting models before proceeding to the test derivation phase.
\newline For the description of the approach, only small examples from a telecommunications case study are provided. & Approach does not present an evaluation. Only for a simpler description, the approach is implemented on a simplified version of a drink vending machine controller.\\
\hline
4b) & Since the approach does not offer any evaluation, only the benefits and shortcomings of the approach have been shared as below:
\newline - The approach provides a solution for tracing only functional requirements and supports only online testing.
\newline - The approach provided automation of the transitions between the phases of the process, allowing to have a fast feed-back loop for testing and debugging specifications or the implementation of the SUT.
\newline - Many errors have been detected in the early stages of the process, when the system models have been created. The errors were caused mainly by omissions in the models and by misinterpreting the requirements.
\newline - Since the approach is fully automated, the effort in updating the models and performing the testing again was diminished.
 & Since the approach does not offer any evaluation, only the benefits and shortcomings of the approach have been shared as below:
 \newline - Automatically produced traceability matrix leads to several advantages in the software testing process, e.g. it gives a clear functional coverage metrics to the generated test cases, enables to improve the model or test generation strategies, and gives valuable feed-back on the requirements.
 \newline - There is no tool support for tracing-back requirements from tests to models.\\
\hline
\end{longtable}
\end{small}
%\end{scriptsize}
\end{landscape} 
\restoregeometry


\newpage
\section{Conclusion}
\label{sec:Conc} 
This chapter presents two approaches for the testing with system models. Both approaches, \cite{Paper1} and \cite{Paper2}, use requirements traceability in their model-based test generation processes and provide a tool support to generate tests automatically. 

The literature research showed us that the for this topic relevant articles are mostly published between 2000-2009. Although there are a lot of articles which provide approaches based on the links (traces) between requirements and test cases, only a few of them are using system models. Therefore, it is important to differentiate the system model and test model with respect to the model-based testing. However, based on our research, we could not find a sharp difference between the two. Our conclusion is that test models are developed solely for testing while system models can be primarily developed for system development and then used for testing as well. Moreover, with this literature study, we came to the conclusion that it is laborious to find relevant articles with keyword-based search method, because requirements traceability and model-based testing are very comprehensive topics in Software Engineering. 

The first approach studied in this chapter, \cite{Paper1}, describes the MBT process in more detail than the second approach \cite{Paper2} does. But \cite{Paper2} presents a more detailed use case example with more visuals which enables us to understand the steps of the testing process better. However, both approaches lack the detailed explanation on how the test cases are generated via the provided test generation tools. For the generation of test cases, \cite{Paper1} provided Conformiq's Qtronic tool, which is unfortunately not existing anymore, and \cite{Paper2} provided the LEIRIOS Test Designer tool, which is named currently Smartesting. Therefore, we could not find the necessary information on how these tools implement the test generation step in practice. Unfortunately, no evaluation was offered for either approach. All of this makes it difficult for us to make an accurate assessment between these two tools/approaches. In our opinion, if we had to choose one of this approaches, it would make more sense to choose the second approach \cite{Paper2}, as LEIRIOS Test Designer tool (Smartesting) is still on the market. It has probably evolved over the ten years as well.

Finally, both approaches presented in this chapter showed us that testing with system models enables to increase the possibility of finding errors, reduce the testing time, and improve the test quality using requirements traceability.

\chapter{Topic 8}\label{sec:topic_8}

%
% Hinweis:
%
% Die Tabellenanordnung, etc muss noch überprüft werden.
% Teilweise sind Figures auf der falschen Seite, trotz [h!]
%

\chapter{Testing Non-Functional Requirements with Risk Analysis}\label{sec:topic_9}

% Andre Meyering

\section{Introduction}

In contrast to functional requirements (FRs) that describe the program's functionality, i.e. how it processes data and user input, non-functional requirements (NFRs) describe constraints that the program must adhere to~\cite{SWEBOK}.
Parts of NFRs are performance and security requirements which can directly affect the end-user but also maintainability requirements which are more important to development teams.
While there are lot of resources about testing FRs in the form of unit-, integration- and end-to-end-tests, no common testing framework or guidance for testing NFRs exists.

For this reason, we look into two approaches \cite{ZouPavlovski2008} and \cite{Lagerstedt2014}.
While \cite{ZouPavlovski2008} was given in advance by the advisors of the seminar, \cite{Lagerstedt2014} was found through a literature search, which is described in \autoref{sec:9_literature}.
Each approach will be described and applied to an example in the form of movie management software in the respective \autoref{sec:9_approach_1} and \autoref{sec:9_approach_2}. 

Both approaches are compared in \autoref{sec:9_comparison} by using a synthesis matrix. We will look at which NFRs are covered and how they are tested.
The results of this chapter will be summarized and concluded in \autoref{sec:9_conclusion}.

Please refer to the glossary for the following terms used throughout this chapter:
test case, use case, NFR.



%%%%%%%%%%%%%%%%%%%%%%%%%%%%%%%%%%%%%%%%%%%%%%%%%%%%%%%%%%%%%%%%%%%%%%%%%%%%%%
%%%%%%%%%%%%%%%%%%%%%%%%%%%%%%%%%%%%%%%%%%%%%%%%%%%%%%%%%%%%%%%%%%%%%%%%%%%%%%

\section{Literature search} \label{sec:9_literature}

The starting point for the literature search was the paper given to us~\cite{ZouPavlovski2008}.
Based on this paper, we formulated the central research question: 

\hspace*{0.5cm}\enquote{\textit{Which approaches for testing non-functional requirements systematically\newline \hspace*{0.7cm}with risk analysis exist?}}.

We focused on finding articles that covered the three most important keywords and phrases for the topic: \textit{testing}, \textit{non-functional requirements} and \textit{risk analysis}.
A quick search using these three phrases resulted in IEEE Xplore having the most promising results, whereas ACM\footnote{\url{https://dl.acm.org/}} only showed a few.
Because the given article \cite{ZouPavlovski2008} can also be found on IEEE~Xplore\footnote{\url{https://ieeexplore.ieee.org/Xplore/home.jsp}}, we focused our search onto that site but still looked at ACM.

To be able to evaluate the relevance of papers found during the literature search, we defined three relevance criteria:

\begin{enumerate}
	\item Does the article cover non-functional requirements? They must not only  be mentioned as a side note next to functional requirements.
	\item Does the article combine risk analysis with tests?
	\item Does the article cover \textit{testing} of non-functional requirements?
\end{enumerate}

Even though these relevance criteria basically only cover the research question, they filter out most non-relevant papers as we will see later on.

% https://ieeexplore.ieee.org/document/4578345
The search was carried out by using forward and backward snowballing as well as by using search terms with combinatorial modifiers.
Only two papers reference \cite{ZouPavlovski2008} according to its IEEE~Xplore site, both of which cover functional but not non-functional requirements.
The paper itself references 23 papers.
Of those papers, only few covered the first criterion and none covered the third criterion.
\cite{ZouPavlovski2008} itself does not cover risk analysis as a main research topic but only covers it in a side note (see \autoref{sec:9_approach_1}).
This is why search-term based search was performed using the key terms: non-functional requirements, testing and risk analysis.

\begin{small}
	\centering
	\begin{longtable}[h]{p{0.08\textwidth}|c|p{0.58\textwidth}|p{0.12\textwidth}}
		\caption{Term based search results}
		\label{tbl:search_topic_9_andre}
		\setlength{\tabcolsep}{1em}\\    %%%%<===
		\toprule
		\textbf{Source} & \textbf{Date} & \textbf{Search query and restrictions} & \textbf{\#Results (relevant)} \\
		
		\midrule
		IEEE Xplore & 2020-11-11 & \texttt{"{}non-functional requirements"{} AND testing AND "{}risk analysis"{}} & 3 (0) \\
		
		\midrule
		IEEE Xplore & 2020-11-11 & \texttt{non-functional requirements AND testing AND risk analysis} & 12 (0) \\
		
		\midrule
		IEEE Xplore & 2020-11-11 & \texttt{risk AND "non-functional" and test} & 23 (2) \\
		
		\midrule
		IEEE Xplore & 2020-11-29 & \texttt{(("{}Abstract"{}:nonfunctional requirements) AND "{}Abstract"{}:Test) AND "Full Text \& Metadata":"{}risk analysis"{})} & 2 (1) \\
		
		\midrule
		IEEE Xplore & 2020-11-29 & \texttt{(("{}Abstract"{}:non-functional requirements) AND "{}Abstract"{}:Test) AND "Full Text \& Metadata":"{}risk analysis"{})} & 3 (0) \\
		
		\midrule
		ACM & 2020-11-29 & \texttt{[Abstract: "{}risk"{}] AND [Abstract: test*] AND [Abstract: "{}non functional"{}]} & 4 (1) \\
		
		\midrule
		ACM & 2020-11-29 & \texttt{[Abstract: test] AND [Abstract: "{}non functional"{}] AND [[Full Text: "{}risk analysis"{}] OR [Full Text: "{}risk"{}]]} & 20 (1) \\
		\bottomrule
	\end{longtable}
\end{small}

\autoref{tbl:search_topic_9_andre} lists an excerpt of the term-based search.
Listed are only those searches that returned promising results or highlight issues I encountered during the search.
It can be seen that, if all keywords are combined using the \texttt{AND} operator with the default restrictions, no relevant results were returned.
After a feedback from one advisor, the search was changed so that \enquote{non-functional requirements} and \enquote{testing} were expected in the paper's abstract, but \enquote{risk analysis} was searched for in \textit{all} metadata including the full text.
It turned out that no papers were found which mentioned risk analysis as well as the other two keywords in their abstract.

We also discovered that the spelling of the term \enquote{non-functional} had a huge impact on the results returned by IEEE~Xplore.

After this initial search, the resulting papers were evaluated and one paper was chosen. The papers to chose from included:

\begin{itemize}
	\item \enquote{Scenario-Based Assessment of non-functional Requirements } \cite{Andre_Search_1}
	\item \enquote{Alignment of requirements specification and testing: A systematic mapping study} \cite{Andre_Search_2}
	\item \enquote{Using Automated Tests for Communicating and Verifying Non-functional Requirements} \cite{Lagerstedt2014}
\end{itemize}


\textit{Scenario-Based Assessment of non-functional Requirements} covers all three criteria we defined at the start of our search. However, it limits itself to complex socio-technical systems and only looks at one non-functional requirement, which is the system's performance.  It limits itself to the evaluation of the reliability of certain aspects of software which interacts with humans to calculate the risk of human errors occurring.
This is done by implementing scenarios--hence the title \enquote{scenario-based assessment}. The testing aspect of this paper is limited to human-interactions whose risks are evaluated. If scenarios fail this risk assessment, then so will the test.

\textit{Alignment of requirements specification and testing: A systematic mapping study} is about papers that cover non-functional requirements. It is a study about such papers and lists approaches that are used to test NFRs. Some of which are mentioned in other chapters of this paper. However, none cover risk analysis. The paper itself does not give much insight into testing non-functional requirements itself.

The chosen article which we will further evaluate in the following sections is \textit{Using Automated Tests for Communicating and Verifying Non-functional Requirements}. It covers the non-functional requirement \enquote{maintainability} and how it can be tested. It further explains it by using practical examples.
However, the chosen paper does not cover risk-analysis. Since no paper could be found that covers all criteria, we were advised to focus on the testing of non-functional requirements and leave out risk-analysis.


%%%%%%%%%%%%%%%%%%%%%%%%%%%%%%%%%%%%%%%%%%%%%%%%%%%%%%%%%%%%%%%%%%%%%%%%%%%%%%
%%%%%%%%%%%%%%%%%%%%%%%%%%%%%%%%%%%%%%%%%%%%%%%%%%%%%%%%%%%%%%%%%%%%%%%%%%%%%%


\section{Approach 1: Control Cases during the Software Development Life-Cycle } \label{sec:9_approach_1}


\subsection{Description}

In their paper \enquote{Control Cases during the Software Development Life-Cycle}~\cite{ZouPavlovski2008}, J. Zou und C. J. Pavlovski based their work on so called  \enquote{control cases} and \enquote{operating conditions} as tools for modeling and controlling NFRs.
\newpage
\enquote{Control cases} are used as a format to communicate and discuss NFRs between management, requirement engineers, developers, and system users and to define qualitative attributes of the system.

Their work focuses on determining and revealing problems early on, for example bad performance or security risks.
The classic software development life cycle often focuses on these topics too late or not at all.
However, control cases require NFRs to be defined first.
To define them, the paper starts by introducing operating conditions.
Operating conditions model constraints that apply to the system or a specific use case. These constraints are then used to model NFRs, hence the operating condition can be seen as a high level view on NFRs.
Defining such constraints that apply to a certain use case is left to the reader or rather is mentioned as a step of the business process modeling.

Operating conditions can belong to one or more use cases and are not unique to a specific one. Conditions such as \enquote{Transaction Volume Condition: $>$400 concurrent users} can be applied to different use cases~\cite{ZouPavlovski2008} and model exactly that: a condition under which the use case operates.

Control cases--as the name suggests--control the operating conditions and can be used to ensure that they are complied to.
This makes it possible to mitigate business risks which may affect the business if the operating condition and its constraints are violated.
Their paper visualizes the connection between these artifacts using an UML diagram, which can be seen in \autoref{fig:topic_9_appraoch_1_use_case} below.

\begin{figure}[htbp]
	\centering
	\includegraphics[width=0.4\textwidth]{../images/topic_9_approach_1_1.png}
	\caption{Association in Use Cases Modelling~\cite{ZouPavlovski2008}}
	\label{fig:topic_9_appraoch_1_use_case}
\end{figure}

The paper creates such a control case by introducing a fictional example of a traveling agent.
All of the previously mentioned artifacts are created during the \enquote{Business Process Modeling} and are refined throughout the software development life cycle.
This means that operating conditions and control cases are defined together with use cases and can be incorporated together in a use case model.
The paper does this for their fictional example which can be seen in \autoref{fig:topic_9_appraoch_1_2}. 

\begin{figure}[h!]
	\centering
	\includegraphics[width=0.65\textwidth]{../images/topic_9_approach_1_2.png}
	\caption{Use Case Model with Control Cases~\cite{ZouPavlovski2008}}
	\label{fig:topic_9_appraoch_1_2}
\end{figure}

In this graphic, control cases are visualized as shaded ellipses and operating conditions as speedometers, though unspecified by the paper. 
This graphic also emphasizes that operating conditions are not bound to one specific use case but can be applied to different ones.
And the control case is specific to one operating condition.


The reader is guided through all steps of the software development life cycle, so that a control case can be defined which is then used as the basis for a test case. Because the control case is associated to an operating condition, the test case is associated to it transitively as well.
The test case exists to verify that the controls put in place to manage the operating condition are effective.
\newpage
The paper does not give a detailed instruction how to model test cases. It only instructs testers to simulate the operating condition, for example by creating a huge work load on the server.
With this simulation relevant metrics can be extracted that are used to verify the test case.



\subsection{Application}

J. Zou und C. J. Pavlovski focus on creating control cases. This is done for the movie manager example.

We first define one goal of our software:
it must contain a movie list view that has smooth scrolling and can handle a large amount of movies.
This also defines a constraint and therefore our operating condition: we must operate a smooth list view.
If this cannot be accomplished an associated risk may affect the business. The control case bundles all of this in a matrix which is defined as in \autoref{tbl:topic_9_approach_1} on \autopageref{tbl:topic_9_approach_1}.

Based on this control case, developers can start to implement the software. During
testing stage, functional requirements can be tested by basing them on use cases. Non-functional requirements, on the other hand, can be tested by creating tests from control cases.
The tester needs to simulate the operating condition.
In our example above, that would mean to simulate the scrolling condition by creating a huge list of movies.
The steps that must be executed for the test are combined into a test case , such as \autoref{tbl:topic_9_test_case} on \autopageref{tbl:topic_9_test_case}.

By determining the operating condition that is associated to a use case, we were able to create a control case that reflects the NFRs.
Based on the operating condition, we then created a test case that checks if the controls put in place by the control cases are enough to mitigate the business risk.
Following this pattern, tests for non-functional requirements can be created systematically step by step.

%\clearpage

\begin{table}[p!]
	\centering
	\caption{Control Case for Approach 1 of Topic 9}
	\label{tbl:topic_9_approach_1}
	\begin{tabular}{|p{0.95\textwidth}|}\hline
		\textbf{Control Case}: Performance of the movie list view \\ 
		\hline
		\textbf{Control Case ID}: CC-001 \\
		\hline
		\textbf{Operating Condition}: Scrolling Speed Condition \\
		\hline
		\textbf{Description}: The control case describes the \enquote{smoothness} while scrolling through the movie list view. Scrolling must be smooth. If it is not then users may assume bad performance. \\
		\hline
		\textbf{NFR Category}: Performance and Capacity\\
		\hline
		\textbf{Associated Use Cases}: Show movies in list view \\
		\hline
		\textbf{Technical Constraints}: GUI Framework, Operating System (e.g. 32bit system only allows addressing of 4GB main memory) \\
		\hline
		\textbf{Vulnerability}: \newline Unknown number of movies. Users may only have a few or thousands of movies.
		Analyzing movies (or doing other work) must not lead to the movie list view being unresponsive. Having a lot of movies must not make the program run out of memory. \\
		\hline
		\textbf{Threat Source}: None (local software used by one user) \\
		\hline
		\textbf{Operating Condition}: There may be tens of thousands of movies. Assuming that each movie object has a size of 600kB (only meta data and a small thumbnail), loading 20,000 movies would lead up to 12GB of memory usage\tablefootnote{From personal experience by maintaining a media manager. Users regularly report more than 10,000 movies in their database.}. All movies must be represented in a list view.\\
		\hline
		\textbf{Business Risk}:\newline If scrolling is not smooth, the user may switch to other software or leave a bad rating. \\
		\hline
		\textbf{Probability}: medium (likely few users are affected) \\
		\hline
		\textbf{Risk Estimation}: \newline
		low (users with huge databases may accept higher load times or sluggishness in the UI) \\
		\hline
		\textbf{Control}: 
		\begin{enumerate}
			\item Only load visible movies into main memory. Use \enquote{infinite scrolling} techniques. Remove those movies from main memory that are not visible to the user.
			\item Only load the title into main memory. Load other details only if required. This reduces the memory footprint.
		\end{enumerate} \\
		\hline
	\end{tabular}
\end{table}


%
% ACHTUNG
%
% Bitte darauf achten, dass diese Tabelle auch bei Ansatz 1 landet und
% nicht bei Ansatz 2, wie es bei mir der Fall war.
%
\begin{table}[h!]
	\centering
	\caption{Test Case for the movie manager example of topic 9, approach 1}
	\label{tbl:topic_9_test_case}
	\begin{tabular}{|p{0.95\textwidth}|}\hline
		\textbf{Associated Control Case ID:} CC-001\\
		\hline
		\textbf{Test Objectives:} \newline Verify that the movie list view has no visible hiccups when scrolling through the list of movies. \\
		\hline
		\textbf{Preconditions:} Movie manager is up and running.\\
		\hline
		\textbf{Test Steps:} \begin{enumerate}
			\item Create 20.000 movies and load them into the movie manager
			\item Open the movie list view
			\item Scroll through the list of movies
		\end{enumerate} \\
		\hline
		\textbf{Expected Result:} \begin{enumerate}
			\item The end of the list view is reached.
			\item No hiccups while scrolling were visible, i.e. no \enquote{sluggishness}.
		\end{enumerate} \\
		\hline
		\textbf{Notes:} \newline
		The test must be performed on a system that has at most 8 GB of RAM to reflect common end-user hardware. \\
		\hline
		\textbf{Test Result:} Pass / Fail \\
		\hline
										
	\end{tabular}
\end{table}

%%%%%%%%%%%%%%%%%%%%%%%%%%%%%%%%%%%%%%%%%%%%%%%%%%%%%%%%%%%%%%%%%%%%%%%%%%%%%%
%%%%%%%%%%%%%%%%%%%%%%%%%%%%%%%%%%%%%%%%%%%%%%%%%%%%%%%%%%%%%%%%%%%%%%%%%%%%%%
\newpage

\section{Approach 2: Using Automated Tests for Communicating and Verifying Non-functional Requirements} \label{sec:9_approach_2}

\subsection{Description}

In \enquote{Using Automated Tests for Communicating and Verifying Non-functional Requirements}~\cite{Lagerstedt2014}, Robert Lagerstedt describes how testing NFRs can be automated by introducing a tool-based approach.
The author only looks at NFRs in regards to software architecture which affects code quality in the sense of maintainability and security.

By looking at software architecture aspects as NFRs, Lagerstedt describes how software may be written by listing some architectural requirements.
It should not have dependencies from lower code components into higher but only vice versa.
Certain functions must not be called from some components to ensure encapsulation. Some functions may be blacklisted due to security concerns.
All of these requirements are part of the software architecture and therefore a huge part of software quality and maintainability~\cite{Lagerstedt2014}.

These NFRs must be communicated to developers. According to Lagerstedt, this is done by guidelines written by software architects.
The compliance of these guidelines is often verified by different reports. These reports may be written for each code change as part of a code review or by other teams.
Lagerstedt visualizes this in a simple UML diagram as can be seen in
\autoref{fig:topic_9_approach_2_1}.
The graphic uses a rather high distance between the developer and the compliance report on purpose to symbolize that the two are asynchronous, this means that the report is not automated and feedback reaches the developer not immediately.

\begin{figure}[htbp]
	\centering
	\includegraphics[width=0.6\textwidth]{../images/topic_9_approach_2_1.png}
	\caption{The common way of communicating architectural requirements~\cite{Lagerstedt2014}}
	\label{fig:topic_9_approach_2_1}
\end{figure}

This way of communicating guidelines is not very cost-efficient.
Every developer has to read and understand the guidelines. Developers must be re-trained when changes are made or if too many guidelines violations occur, because they have been forgotten.
This is quite time consuming and prone to error.  Creating reports about guideline compliance is time consuming as well.
Furthermore, while code review should be performed for all code changes, mistakes may slip through.

That is why the author proposes automated testing of software architecture NFRs.
This allows a fast tool-based feedback loop in which the developer gets a code review that can be incorporated without other developers having to look out for violations of guidelines.
On top of that, by having this tight feedback loop, developers can learn the guidelines in an iterative way.
Little to no training is required, which saves time to make new guidelines known to all developers.

The guidelines are written as tests. These tests can be included in existing static code analysis tools such as linters and other code checkers. Developers can see the results of such tools.
Furthermore guidelines are communicated to the developer in case of a test failure.
Lagerstedt uses \autoref{fig:topic_9_approach_2_2} to visualize this approach.
Developers get feedback through different tools that the architect extends. Tools such as the editor, compiler or static analysis tools.

\begin{figure}[htbp]
	\centering
	\includegraphics[width=0.7\textwidth]{../images/topic_9_approach_2_2.png}
	\caption{Suggested solution of communicating requirements~\cite{Lagerstedt2014}}
	\label{fig:topic_9_approach_2_2}
\end{figure}
\newpage
According to Lagerstedt's personal experience, a tool based approach is superior to a guideline-only one.
Productivity is increased while the time spent on communicating architectural guidelines is decreased.
The number of non-compliant code is lower for the tool-based approach than for using guidelines and reports only.


\subsection{Application}

The paper works with architectural NFRs but does not explain how those can be modeled.
To be able to apply the approach, we introduce another system function to the movie manager example that is listed in \autoref{tbl:topic_9_application_approach_2}.
This system function and the following NFRs are based on personal experience in maintaining an open-source media manager.

\begin{table}[h!]
	\centering
	\caption{New system function for application of approach 2 of topic 9}
	\label{tbl:topic_9_application_approach_2}
	\begin{tabular}{r|p{0.7\textwidth}}
		\hline
		\textbf{Name}          & Export the movie to HTML \\ \hline
		\textbf{Description}   & An existing movie is exported to a single HTML file which can be viewed in any modern web browser \\ \hline
		\textbf{Precondition}  & Movie exists \\ \hline
		\textbf{Input}         & Movie details \\ \hline
		\textbf{Postcondition} & HTML file exists with the movie's contents \\ \hline
		\textbf{Output}        & HTML file \\ \hline
	\end{tabular}
\end{table}

In \autoref{tbl:topic_9_approach_2_nfr} on \autopageref{tbl:topic_9_approach_2_nfr}, two NFRs are listed which were created for the system function in \autoref{tbl:topic_9_application_approach_2}. 
These two NFRs are based on personal experience.
Both are transformed into pseudo code so that the NFRs can be executed automatically as part of the code review.

While these two NFRs can be written as guidelines, especially point two may be violated and may slip through code review. Violating point two may result in security issues or at least in unexpected behavior if the HTML contains unescaped characters.
\newpage
\begin{longtable}{p{0.05\textwidth}|p{0.24\textwidth}|p{0.65\textwidth}}
	\caption{NFRs for the application of approach 2 of topic 9}
	\label{tbl:topic_9_approach_2_nfr}
	\setlength{\tabcolsep}{1em} \\
	\hline    %%%%<===
	\textbf{No.} & \textbf{NFR} & \textbf{Explanation} \\
	\hline
	1 & IMDb IDs are encapsulated in a class &
	All IMDb IDs have a certain format. They start with the string \enquote{tt} and end with 7-8 numbers. The ID must be validated which cannot be ensured by using a simple string. This is why an encapsulation in a class is required.
	
	Furthermore the programming language's type system can help to identify conversion bugs as well.
	\newline \newline \textit{Implementation in pseudo code}
	\begin{lstlisting}
for each $variable in $source:
 if $variable.startsWith("imdb") then
  if typeof($variable) != "ImdbId" then
   throw new Exception("Wrong class")
	\end{lstlisting}\\
	\hline
	
	2 & Exported strings are escaped &
	This is a security concern and can be implemented in different ways. We assume that an HTML-exporter was created which takes a movie object as an argument.
	This object may contain texts which contain HTML elements themselves.
	These elements need to be escaped. To ensure this NFR, all strings must be run through a certain function which escapes strings.
	
	Because this may be missed by the developer, a new string-subclass is introduced which escapes its input automatically, e.g. \texttt{EscapedString}.
	Only this string class may be used in the HTML exporter.
	\newline \newline \textit{Implementation in pseudo code}
	
	\begin{lstlisting}
for each $functionCall in $HTMLExporter:
 if $functionCall == "writeText" then
  $arg = argument of($functionCall);
  if typeof($arg) != "EscapedString" then
   throw new Exception("Wrong class")
	\end{lstlisting}

\textit{Note:} We assume that \texttt{writeText} is a method of a generic HTML-class which the HTML-exporter uses itself. We assume that the method cannot be changed to accept another argument type. Otherwise the language's type checker could already be able to find this issue.\\

\hline
\end{longtable}


%%%%%%%%%%%%%%%%%%%%%%%%%%%%%%%%%%%%%%%%%%%%%%%%%%%%%%%%%%%%%%%%%%%%%%%%%%%%%%
%%%%%%%%%%%%%%%%%%%%%%%%%%%%%%%%%%%%%%%%%%%%%%%%%%%%%%%%%%%%%%%%%%%%%%%%%%%%%%
%\clearpage
\section{Comparison} \label{sec:9_comparison}

For an improved comparison of these two approaches, a synthesis matrix is provided which references the questions from \autoref{sec:intruduction1.3}:

%\begin{enumerate}
%	\item Description of the approach (What does the approach do?)
%	\begin{enumerate}
%		\item Which artifacts and relations between artifacts are used in this approach? Which artifacts are created in the course of the approach? How are the artifacts characterized?
%		\item What is required and/or input for the application of the approach?
%		\item Which steps does the approach consist of? Which information is used in which step and how? What are the results of the individual steps?
%	\end{enumerate}
%	\item Benefits of the approach (Whom does the approach help and how?)
%	\begin{enumerate}
%		\item Which usage scenarios are supported by the approach?
%		\item Which stakeholders are supported by the usage scenarios?
%		\item Which knowledge areas from SWEBOK can be assigned to the usage scenarios?
%	\end{enumerate}
%	\item Tool support for the approach (What tool support is available?)
%	\begin{enumerate}
%		\item What kind of tool support is provided for the approach?
%		\item Which steps of the approach are automated by a tool? Which steps are supported by a tool, but still have to be executed manually? Which steps are not supported by a tool?
%	\end{enumerate}
%	\item Quality of the approach (How well does the approach work?)
%	\begin{enumerate}
%		\item How was the approach evaluated?
%		\item What are the (main) results of the evaluation?
%	\end{enumerate}
%\end{enumerate}


\begin{small}
\begin{longtable}[H]{p{0.03\linewidth}|p{0.455\linewidth}|p{0.455\linewidth}}
	\hline
	\textbf{No.} & \textbf{Approach 1 \cite{ZouPavlovski2008}} & \textbf{Approach 2 \cite{Lagerstedt2014}} \\
	\hline
	1a) & 
	Operating conditions are formed that work under specific use cases.
	These, on the other, hand are controlled by control cases and can be operated under them.
	Control cases describe the business risks in case that the operating condition cannot be fulfilled.
	Because use cases and control cases are tightly connected to each other, they can be modeled in one consolidated model.
	& 
	
	Coding guidelines are written and transformed into tests that can be used by tools in code review.
	These point out issues that the developer can fix.
	Guidelines are characterized by the fact that they describe the code architecture.
	\\
	\hline
	1b) & 
	There are no preconditions because we start defining control cases at the beginning of the software development process, for example at the \enquote{Business Process Modelling}-step.
	
	&
	The guidelines cover code architecture. They must be transformable into automated tests (e.g. by a static code analyzer).
	
	\\
	\hline
	
	1c) &
	Preconditions/Constraints must be extracted from which NFRs are created, e.g. performance or security constraints.
	These constraints are modeled by operating conditions for which control cases are created.
	Their purpose is to mitigate business risk which is essentially the failure to fulfill the operating condition.
	For each control case a test case is added that checks if the controls put in place by the control case are effective.
	The test case basically recreates the operating condition, for example by using stress testing.
	
	&
	Code guidelines such as naming conventions or prohibited function-calls are defined.
	These are transformed into automated tests that can be executed by the developer (i.e. a tool based approach).
	The exact process is not explained and it is left to the reader how this may be implemented.
	It is only pointed out that existing tools such as compilers or static code analysis tools can be extended and used.
	\\
	\hline
	
	2a) & 
	
	Early modeling of non-functional requirements. Being able to control requirements throughout the whole software development life cycle.
	
	&
	Maintainability, quality and security of the code base can be hold up to standards and may even be improved by giving automated feedback that points out NFRs which are violated by the developer.
	
	\\
	\hline
	2b) & 
	Management, Requirements Engineer, Developers, Testers
	&
	Developer / test writer, Software Architect
	\\
	\hline
	
	2c &
	Software Requirements (functional and non-functional requirements, acceptance tests), Software Testing (model based techniques)
	&
	Software Testing (Software Testing Tools, Test Techniques), Software Maintenance (Software Maintenance Tools) \\
	\hline
	3a) & 
	No tool support for generating \enquote{Control Case}-Boxes and other artifacts &
	Existing static code analysis tools (e.g. linters), which can be extended by further tests.
	\\
	3b) & 
	No automation is done. Automation is only proposed as another step which can be implemented, e.g. through code generation with SysML.
	&
	Only code testing is performed automatically.
	And only tests for NFRs which were extracted from the software architects guidelines and that were transformed into automated tests.
	Those tests can be executed automatically during code review, e.g. by a continuous-integration service which tests each code change. Writing the tests is still a manual job.
	\\
	\hline
	4a)
	&
	The approach was explained by creating a fictional example and going through all steps of the software development life cycle by extending the example. No evaluation was performed, though.
	
	&
	No evaluation was performed. The conclusion, i.e. success of the approach, is based on personal experience only.
	\\
	\hline
	4b) &
	N/A &
	
	Based on his experience in both small and large organizations, Lagerstedt concludes that automated verification of non-functional requirements by using tool-chain feedback is superior to classic guidelines that need to be checked by humans. By evaluation of his prior experience, he concludes increased productivity and a decrease in time spent on communicating architectural requirements. \\
	\hline
\end{longtable}
\end{small}


If we compare the two papers using the synthesis matrix above, we notice that they do not share a lot. That is not surprising: the second paper is very specific and only deals with architectural NFRs in code. The first paper, on the other hand, can be applied to different NFRs, not limiting itself to a specific one.
Only the first paper mentions risk analysis but only as part of a control case.

Both papers do not give specific instructions how test cases can be modeled.
While the first paper only says to \enquote{simulate the operating condition}~\cite{ZouPavlovski2008}, it leaves out details.
For example security NFRs are explicitly mentioned but it is left out how an operating condition for that NFR can be simulated.
Also the example test case from the paper is essentially a stress test.
The second paper leaves it to the reader to develop automated tests and only mentions that static code analysis tools can be used.

While the second paper talks about test automation, it does not talk about creating tests automatically but rather about running them automatically~\citealp{Lagerstedt2014}.
The first paper does not include any automation step at all.
Neither for creating test cases automatically nor for running them.

Both do not include an evaluation of their results besides personal experience.
The first article states no evaluation at all and only discusses the approach for defining the control case.


%%%%%%%%%%%%%%%%%%%%%%%%%%%%%%%%%%%%%%%%%%%%%%%%%%%%%%%%%%%%%%%%%%%%%%%%%%%%%%
%%%%%%%%%%%%%%%%%%%%%%%%%%%%%%%%%%%%%%%%%%%%%%%%%%%%%%%%%%%%%%%%%%%%%%%%%%%%%%
\newpage
\section{Conclusion} \label{sec:9_conclusion}

Both articles deal with NFRs.
While \cite{ZouPavlovski2008} describes how these can be defined and controlled, it does not specify a way to test them except for simulating the operating condition.
In the same way there is no description of how the business risk affects the test case except for defining the test-priority.
However, it explains in great detail how control cases and operating conditions can be defined and how they interact with use cases and functional requirements, which raised my interest in the overall topic of testing NFRs.
But the lack of detailed explanation for test case creation makes it difficult for me to assess the usefulness of the approach. After reading the paper I may know how to model NFRs with operating conditions but still wonder how they can be properly tested.

\cite{Lagerstedt2014} on the other hand leaves it to software architects to define NFRs.
The paper only uses architectural NFRs that exist as code conventions and other guidelines.
The author describes why having automated tests is a necessity of software development in regards to cost efficiency and how it mitigates human error during code review which can be seen as a risk to code maintainability.
This corresponds to my personal experience. 
By using a code formatter, the amount of formatting related review comments went down to zero.
By introducing a new linter rule, I was able to automatically fix company branding issues in product messages which none of my colleagues were even aware of.
I can therefore only emphasize that communication of NFRs is more effective and  efficient when a tool based approach is used.


Finally, both articles mention risk analysis only as a side note, if mentioned at all.
It is left to the reader where risks are mitigated.
The conclusion is that NFRs with higher risks need to be paid more attention to by giving the tests higher priority.


\chapter{Testing nonfunctional requirements with aspects}\label{sec:topic_10}

\section{Introduction}

Software testing ensures that specified requirements, functional and non-functional, are met by the implementation. Non-functional requirements are especially hard to test because they do not describe what the software does, but how and to what extend of quality it does it. Hence the also customary term quality requirements. They pose numerous challenges for software testing and software quality due to their system-wide effects and often crosscutting nature. They do not only concern one part of the software and its source code, but multiple or even the entire system. For instance, the memory requirement: “The system only uses two gigabytes of main memory at most at all times.” has a restrictive influence on other requirements and system functionality like performance requirements. This impedes the observance of the widely propagated principle of separation of concerns, introduced by Parnas \cite{Parnas} and Dijkstra \cite{Dijkstra}, throughout development and testing.\\
\\
By virtue of the afore-mentioned problems, there are only few tools and systematic approaches for testing non-functional requirements. Frequently, they are evaluated subjectively, resulting in a loss of traceability between test code, source code and requirements. Aspect orientation is a technique that can be harnessed for testing non-functional requirements. It is a programming paradigm, aiming to modularise such requirements (in the context of AOP, they are called system-level concerns in contrast to core-level concerns), similar to the modularisation of functional requirements using objects. An aspect modularises a system-level concern and thereby represents a system-wide functionality, accessible to multiple classes and other parts of the software. Let us consider a common application for AOP: Listing \ref{logging} shows a simple logging aspect in C++. It consists of three main components:
\begin{itemize}
\item Joinpoint: The point in the program code, the aspect is executed. It can be before, after or around something (here: before).
\item Advice: Associates the joinpoint with an activity (here: a printf statement). 
\item Pointcut: Selects a suitable joinpoint out of the set of all joinpoints and associates it with a method or function (here: \%::\%()).
\end{itemize}

\lstset {language=C++}
\begin{lstlisting}[caption={\textbf{Logging Aspect in C++.}}, label=logging]
aspect LoggingAspect{
	public:
		pointcut logMethods() = call("% %::%()");
		advice logMethods() : before(){
			printf("> Enter: %s\n", JoinPoint::signature());
		}
};
\end{lstlisting}

\newpage
In this example, the printf logging statement is executed each time before the \%::\%() method is called. Based on these principles, many possibilities for systematic and automated testing are conceivable.\\
\\
In the following, the execution and results of a literature search based on a given article are presented in Section \ref{lit}. Subsequently, the given article , which assesses the use of aspects for testing non-functional requirements as well as the suitability of certain non-functional requirements for aspect-oriented testing is described and applied to an example in Section \ref{given}. Likewise, for a selected article found using the literature search in Section \ref{found}.  It expands upon the basic ideas of the first article and partially automates the creation of test aspects. Thereafter, the results of a literature comparison are presented in Section \ref{compare}. Finally, in Section \ref{feierabend} both approaches as well as the general concept of testing using aspects are evaluated. 

\section{ Literature Search} \label{lit}

To find relevant literature, a literature research based on the following search question was conducted: \textbf{\textit{Which approaches for systematic creation of tests (for non-functional requirements) using aspects exists?}} Both forward and backward snowballing proceeding from the given article as well as a term-based search were carried out. The used terms were test, aspects, AOP and aspect-oriented. Restrictions to reduce the number of hits are described in Table \ref{restrict}. First, the upper term from Table \ref{restrict} with everything in the publication title was used, then the lower term from Table \ref{restrict} with test in the title, aspects in the abstract and AOP and aspect-oriented in all meta data. Source platforms for the search were IEEE and ACM. IEEE because the given article as well as articles referencing it can be found here. ACM because there are many available and peer-reviewed articles differing from IEEE. Content-based relevance criteria were derived directly from the search question:

\begin{itemize}
\item The article must cover the creation of tests using aspects because this is the main topic of the given article. Furthermore, the criterion is used to exclude articles covering testing of aspect-oriented software with conventional methods.
\item The article must describe systematic approaches for the creation of tests, since this is the superordinate topic of the seminar. 
\item The article must describe test methods for non-functional requirements, as this is the subtopic of the given article.
\end{itemize}

In addition to these content-based criteria, there were some rather soft criteria:
\begin{itemize}
\item The article must be from different authors.
\item The article must be written in English or German language.
\end{itemize}

\begin{table}[h]
\caption{\textbf{Search Terms, Restrictions and Sources.}}
\begin{tabular}{|p{6.5cm}|p{4.5cm}|p{2cm}|}
\hline
\textbf{Term} & \textbf{Restrictions} & \textbf{Sources}\\
\hline
\multirow{2}{8cm}{\textbf{test} AND (\textbf{aspects} OR \textbf{AOP} OR \textbf{aspect-oriented})} & \tabitem all in title  &  \tabitem IEEE\\
& \quad & \tabitem ACM \\
\hline
\multirow{2}{8cm}{\textbf{test} AND \textbf{aspects} AND \textbf{AOP} AND \textbf{aspect-oriented}} & \tabitem test in title  &  \tabitem IEEE\\
& \tabitem aspects in abstract & \tabitem ACM \\
\hline
\end{tabular}
\label{restrict}
\end{table}

Table \ref{doc} shows the execution and results of the literature search. For further selection, the relevant articles were divided into three categories:

\begin{itemize}
\item overviews of testing using aspects,
\item examples for testing using aspects,
\item improvement or monitoring of conventional (unit) test using aspects. 
\end{itemize}

The given article can be classified between overview and example. Via the literature search, I wanted to find an article providing automation and tool support, seen as the given article has shortcomings in that regard. Moreover, all relevance criteria must be fulfilled and it would be beneficial if the article covers at least two out of three classification categories.\\
\\
Many articles violate the first criteria, because they cover the testing of aspect-oriented software without using aspects. Some articles were not chosen because they just cover one classification category, therefore not really adding new information and approaches.\\
\\
The article that stood out was Duclos et al.: “ACRE: An Automated Aspect Creator for Testing C++ Applications” \cite{Duclos} because it covers all categories and satisfies all criteria. It includes an extensive general section followed by the description of an approach for automated creation of test aspects on an example to improve or replace conventional tests. Since the article was found using forward snowballing, it directly references the given article and proposes solutions for problems the latter raises. This article was selected.

\newpage
\newgeometry{margin=1cm}
\begin{landscape}

\begin{table}
\caption{\textbf{Literature Research Documentation.}}
\begin{longtable}{|p{1.3cm}|p{1.8cm}|>{\raggedright}p{2.3cm}|>{\raggedright}p{6cm}|p{1.5cm}|p{1.5cm}|p{1.2cm}|p{2cm}|}

\hline
\textbf{Source} & \textbf{Date} & \textbf{Restrictions} & \textbf{Term} & \textbf{Results} & \textbf{Relevant} & \textbf{Used} & \textbf{Comments}\\
\hline
IEEE$^*$ & 11.11.2020 & none & forward snowballing & 10 & 5 & $\nabla$ & -\\
\hline
IEEE$^*$ & 11.11.2020 & none & backward snowballing & 22 & 5 & none & -\\
\hline
IEEE$^*$ & 11.11.2020 & none & "All Metadata":test AND ("All Metadata": aspects OR "All Metadata":AOP OR "All Metadata":aspect-oriented) & 220,605 & ? & none & too general, not considered\\
\hline
ACM$^\dagger$ & 11.11.2020 & none & [All: test] AND [[All: aspects] OR [All: aop] OR [All: aspect-oriented]] & 172,120 & ? & none & too general, not considered\\
\hline
IEEE$^*$ & 11.11.2020 & title &"Document Title":test AND ("Document Title": aspects OR "Document Title":AOP OR "Document Title":aspect-oriented) & 158 & 11 & none & first 50 considered\\
\hline
ACM$^\dagger$ & 12.11.2020 & title &[Publication Title: test] AND [[Publication Title: aspects] OR [Publication Title: aop] OR [Publication Title: aspect-oriented]] & 311 & 4 & none & first 50 considered\\
\hline
IEEE$^*$ & 12.11.2020 & test: title, aspects: abstract &((("Document Title":test) AND "Abstract":aspects) AND "All Metadata":AOP) AND "All Metadata":aspect-oriented & 33 & 16 & none & -\\
\hline
ACM$^\dagger$ & 12.11.2020 & test: title, aspects: abstract &[Publication Title: test] AND [Abstract: aspects] AND [All: aop] AND [All: aspect-oriented] & 31 & 6 & none & -\\
\hline
\end{longtable}
\textit{\qquad \qquad \qquad \qquad \qquad *: https://ieeexplore.ieee.org/Xplore/home.jsp \quad $\dagger$: https://dl.acm.org/ \quad $\nabla$: \cite{Duclos}}
\label{doc}
\end{table}

\end{landscape}
\restoregeometry

\newpage
\section{Metsä et al.: “Testing Non-Functional Requirements with Aspects: An Industrial Case Study” } \label{given}

\subsection{Description}
The research article “Testing Non-Functional Requirements with Aspects: An Industrial Case Study” by Jani Metsä of Nokia in association with Mika Katara and Tommi Mikkonen \cite{Metsa} of Tampere University of Technology was published in 2007 as part of the Seventh International Conference on Quality Software. According to the authors, the main goal of software testing is to ensure that requirements are met by the implementation. There are already several systematic approaches for testing functional requirements, but only few for testing non-functional requirements. To tackle this problem, they turned to aspect-oriented programming as a potential testing technique. In their paper, they try to answer the following research questions:

\begin{enumerate}
\item To what extent can aspect-oriented techniques be harnessed to test non-functional requirements?
\item Which non-functional requirements lend themselves for testing with aspects?
\end{enumerate}

The authors conducted a case study, analysing 150 requirements of an existing industrial embedded system (a quality verification software for mobile phones running Symbian OS). They were able to identify 16 crosscutting non-functional requirements for which seven test aspects were formulated and implemented. In detail, in the first step, a set of provided system requirements or characteristics are analysed and corresponding non-functional requirements (which might be crosscutting) derived. One non-functional requirement is derived from one or more system requirements, a system requirement can comprise multiple non-functional requirements. In a second step, the non-functional requirements are categorised (performance, robustness, …), and corresponding testing objectives are derived. One testing objective is derived from one or more non-functional requirements. Finally, the test aspects can be formulated based on the testing objectives and non-functional requirements categories. One test aspect can comprise multiple testing objectives.\\  
\\
The approach proved to be easy and test coverage could be increased significantly. The use of aspects enables non-invasive testing throughout the software lifecycle. Furthermore, separation of (test) concerns can be achieved by modularizing crosscutting non-functional requirements. As a result, maintainability, reusability as well as tracing between requirements and test code can be improved.  Based on the experiences gained, the authors concluded that especially system-wide and crosscutting non-functional requirements, such as security, performance, reliability and robustness, are suitable for aspect-oriented testing.
To sum up, the article presents a systematic approach – a systematology - to derive testing objectives and test cases with related test aspects from non-functional requirements. It does not provide automation for any step, hence the main problems the authors identify: the lack of tool support and the need for testing personnel to be firm with aspect-oriented programming.

\newpage
\subsection{Application}

 The non-functional requirements (REQ) are derived from the user task or the corresponding system characteristics (Table \ref{req}). They have system-wide effects (hence affect multiple system-functions) and are of crosscutting nature. For example, REQ5 and REQ6 set performance constraints on REQ7 and REQ8 as well as REQ3; REQ3 and REQ4 set reliability and robustness constraints on each other; REQ9 sets security constraints on REQ1 and REQ2.\\
\\
 Subsequently, the non-functional requirements are categorised and corresponding test objectives can be derived (Table \ref{obj}).\\
\\
Finally, a test aspect for one or multiple requirement categories is formulated (Table \ref{aspects}). These test aspects would then need to be implemented manually using an AOP-Framework (AspectC++ \cite{C++}, AspectJ \cite{J}, …). For instance, the Memory Aspect could work like this: every time the constructor or destructor of an entity class (Movie, Performer, …) is called, a counter will be incremented or decremented by the size of the class.

\begin{table}[h]
\caption{\textbf{Initial System Requirements of the Movie Manager App.}}
\begin{tabular}{|p{7cm}|p{7cm}|}
\hline
\textbf{System Characteristic} & \textbf{Derived Requirement}\\
\hline
\multirow{2}{6.5cm}{Data is managed consistently by the system.} & REQ1: Default values are provided (wherever possible).\\
 & REQ2: Entities are linked consistently (a performer must always be linked to at least one movie).\\ 
\hline  
\multirow{2}{6.5cm}{The system is fault tolerant and able to report faulty behaviour.} & REQ3: The system can recover from hang situations.\\
 & REQ4: The system can identify correct and incorrect system behaviour.\\
\hline 
\multirow{2}{6.5cm}{The system runs on a mobile phone with limited amount of resources and thus has a strict memory footprint.} & REQ5: The system must occupy at maximum W bytes of ROM\\
 & REQ6: The system must occupy at maximum X bytes of RAM.\\
\hline 
\multirow{2}{6.5cm}{The system is fast and responsive.} & REQ7: The system can respond to requests after Y time units after power-up.\\
 & REQ8: All user requests are handled in Z time units.\\
\hline  
The system might hold sensitive data and is therefore secure. & REQ9: The system follows the Android App security best practices (e.g. a password is required for sensitive data changes).\\
\hline
\end{tabular}
\label{req}
\end{table}

\begin{table}
\caption{\textbf{Testing Objectives for Non-Functional Requirements.}}
\begin{tabular}{|p{2cm}|p{8cm}|p{4cm}|}
\hline
\textbf{REQ} & \textbf{Testing Objective} & \textbf{Requirement Category}\\
\hline
REQ1 & TO1: Supervise data consistency and integrity. & Security (Integrity).\\
REQ2 & \quad & \quad \\
\hline
REQ9 & TO2: Check password protection. & Security\\
\hline 
REQ3 & TO3: Generate hang situation. & Robustness\\
\hline 
REQ3 & TO4: Analyse system reliability. & Reliability\\
REQ4 & \quad & \quad \\
\hline 
REQ5 &TO5: Supervise memory consumption. & Performance (Memory)\\
REQ6 & \quad & \quad \\
\hline
REQ7 & TO6: Measure time consumed from power-on to the system being in a responsive state. & Performance\\
\hline 
REQ7 \quad REQ8 &TO7: Measure time consumed on serving requests and executing system-functions. & Performance\\
\quad & \quad & \quad \\
\hline
\end{tabular}
\label{obj}
\end{table}

\begin{table}
\caption{\textbf{Formulated Test Aspects.}}
\begin{tabular}{|p{4cm}|p{8cm}|p{2cm}|}
\hline
\textbf{Test Aspect} & \textbf{Description} & \textbf{Test Objective(s)}\\
\hline
Integrity Aspect & Checks if all specified default values are provided and performers are linked to at least one movie. & TO1\\
\hline
Security Aspect & Tries to execute all data changing system-functions without providing the password. It should ‘t be possible to commit the changes. & TO1, TO2\\
\hline
Robustness Aspect & Generates a request jam to test if the SUT can recover from the hang situation. & TO3\\
\hline
Reliability Aspect & Collects information on SUT states and failures. & TO4\\
\hline
Memory Aspect & Supervise memory consumption by tracking all memory allocations and deallocations. & TO5\\
\hline
Performance Aspect & Measure function execution times. & TO6, TO7\\
\hline
\end{tabular}
\label{aspects}
\end{table}

\newpage
\section{Duclos et al.: “ACRE: An Automated Aspect Creator for Testing C++ Applications” } \label{found}

\subsection{Description}

Etienne Duclos, Sébastien Le Digabel, Yann-Gaël Guéhéneuc and Bram Adams \cite{Duclos} of École Polytechnique de Montréal published their article “ACRE: An Automated Aspect Creator for Testing C++ Applications” in 2013 as part of the 17th European Conference on Software Maintenance and Reengineering. They state that software should be faultless and in accordance with requirements yet testing costs need to be acceptable. Systematic and developer-friendly tools for testing functional and non-functional requirements are required to achieve this goal. However, such tools are sparce, especially for non-functional requirements. The authors review related approaches for testing with aspects, including the approach by Metsä et al., and conclude that high-level tool support and automation are required to solve the problems raised in these publications. To achieve this, they try to answer the following tripartite research question:\\
\\
Can automatically generated test aspects be used to \dots
\begin{enumerate}
\item \dots detect memory leaks?
\item \dots test invariants?  
\item \dots detect interference bugs?
\end{enumerate}

The authors present ACRE (automated aspect creator), a tool that automatically generates test aspect code using a domain specific language. Provided a set of test cases or objectives or a bug report, the DSL statement describing the test aspect can be derived. The type of the bug or the category of the underlying NFR of the test case determines the type of the test aspect that must be chosen in this step. Afterwards, the test aspect is generated automatically based on the DSL description. ACRE takes the entire source code as an input and looks for DSL statements. They are parsed to generate the corresponding test aspects.\\
\\
Using ACRE, the authors were able to detect one error of each type (1. – 3.) in the mathematical optimisation software NOMAD. Thanks to the DSL description of the test aspect, testing personnel does not need to have extensive knowledge about aspect-oriented programming, just about the DSL syntax. Although the creation of test cases from requirements or bug reports is not automated, the DSL and the types of available test aspects provide support in that regard. However, this means that the approach is currently limited to specific use cases (memory leaks, invariants, interference bugs in C++ applications). Nevertheless, it would be easy to extend the available functionalities and transfer the approach to other programming languages with support for aspect-oriented programming. \\
\\
To sum up, the article presents a tool for the automatic generation of test aspects with given test cases formulated in a domain specific language.

\newpage
\subsection{Application}
The non-functional requirements with corresponding test objectives or a bug report must be given, as the approach does not present a way to derive them. Let us consider the same requirements (REQ1-9) and test objectives (TO1-7) as in Table \ref{req} and Table \ref{obj}. Furthermore, a user of the Movie Manager App has filed the following bug report:

\begin{table}[h]
\caption{\textbf{Bug Report.}}
\begin{tabular}{|p{14cm}|}
\hline
\textbf{Summary:} Memory Leak when removing all movies of a performer.\\
\hline
\textbf{Description:} Deleting all Movies linked to a performer leads to the removal of the performer from the performers list, but no memory* is not actually freed.\\
\hline
\textbf{Steps to reproduce:} \begin{enumerate} \item Consider a performer with one linked movie. \item Delete the linked Movie. \end{enumerate}\\
\hline
\textbf{Actual results:} The performer and movie are removed from the respective lists, but no memory is freed.\\
\hline
\textbf{Expected results:} The performer and movie are removed from the respective lists and both objects are no longer in memory.\\ 
\hline
\end{tabular}
\textit{(*For the purpose of this example, let us consider all data is kept in main memory.)}
\label{bug}
\end{table}


Based on the given test objectives or the bug report, the test aspects must be formulated manually using a domain-specific language. However, the DSL provides certain types of Aspects (Counter, Logging, Timing, Checking) and guidelines that help formulate the test cases. Table \ref{aspect-test} shows the aspect types corresponding to the given test objectives and test aspects of approach 1. The DSL formulation to test for memory leaks in the  Performer class  could look like this:

\lstset {language=C++}
\begin{lstlisting}[caption={\textbf{DSL Statement For Counter Aspect}}, label=dsl]
// // name: MemoryAspect              
// // type: Counter                           
// // className: Performer    
// // namespace: MovieManager       
\end{lstlisting}

The parameters \textit{'name'} and \textit{'type'} are required and determine the name and type of the test aspect. 'className' and 'namespace' are optional and specify, which class are concerned by the aspect. By default, the file in which the DSL statement was found is searched for namespaces and the name of the file is considered to be the name of the class.  It must be placed somewhere within the source code, for example above or below the tested function or class or in a separate file. Afterwards, ACRE takes the entire source code as an input, looks for DSL statements (always starting with // //) and generates the corresponding test aspects automatically. For the example above, ACRE generates the memory aspect for the Performer class as shown in Listing \ref{mem}. In essence, the counter aspect initialises a static counter variable that is incremented each time the Performer constructor is called and decremented each time the destructor is called. After the execution of the main method, the final count as well as a warning in case of a memory leak (counter $>$ 0) are printed. In this manner, a class causing a memory leak can be detected. The timing and logging aspects are quite similar to the counter aspect. The checking aspect is more complex and allows the declaration of variables as well as the use of for-loops, while-loops and if-clauses. Note that the timing aspect in its current form is used for interference bug testing, hence it measures access times to variables. However, a timing aspect for measuring function call times could easily be implemented by starting a timer before each function call and stopping it after each function call in the advice of the aspect.

\begin{table}[h]
\caption{\textbf{Aspect Types for Given Test Objectives.}}
\begin{tabular}{|>{\raggedright}p{2cm}|p{8cm}|p{1cm}|p{2cm}|}
\hline
\textbf{Test \quad Aspect} & \textbf{Description} & \textbf{Test Objective(s)} & \textbf{Aspect Type}\\
\hline
Integrity Aspect & Checks if all specified default values are provided and performers are linked to at least one movie. & TO1 & Checking \\
\hline
Security Aspect & Tries to execute all data changing system-functions without providing the password. It should ‘t be possible to commit the changes. & TO1, TO2 & Checking\\
\hline
Robustness Aspect & Generates a request jam to test if the SUT can recover from the hang situation. & TO3 & Checking, Logging\\
\hline
Reliability Aspect & Collects information on SUT states and failures. & TO4 & Logging\\
\hline
Memory Aspect & Supervise memory consumption by tracking all memory allocations and deallocations. & TO5 & Counting\\
\hline
Performance Aspect & Measure function execution times. & TO6, TO7 & Timing (not in current form)\\
\hline
\end{tabular}
\label{aspect-test}
\end{table}

\lstset {language=C++}
\begin{lstlisting}[caption={\textbf{Generated Counter Aspect Code.}}, label=mem]
aspect MemoryAspect{
	public static int _Eval_Performer = 0;

	pointcut Eval_Performer() = "MovieManager::Performer";
	advice Eval_Performer() : slice struct{
		class Eval_Performer{
			public:
				// constructor -> increment
				Eval_Performer(){
					MemoryAspect::_EvalPerformer++;
				}
				// destructor -> decrement
				~Eval_Performer(){
					MemoryAspect::_EvalPerformer--;
				} 
		} 
	};

	// print counter (and warning)
	advice execution (main(...)) : after () {
		printf("Final count of Eval_Performer: \%d\n",  _Eval_Performer)
		if(_Eval_Performer > 0)
			printf("Memory Leak! \n")
	}
};
\end{lstlisting}

\newpage
\section{Comparison} \label{compare}

The approaches have been compared using a set of synthesis questions, as shown in Table \ref{syn}.\\
\\
Both approaches offer the same benefits: using test aspects, it is possible to test system-wide crosscutting non-functional requirements in a non-invasive way (without modifying or instrumentalising the source code). As a result, modularity, maintainability reusability as well as traceability of requirements to corresponding tests can be improved. Multiple stakeholders can benefit from both approaches, in particular testing and maintenance personnel. Correspondingly, both approaches use and produce similar artifacts: test objectives are derived from requirements and ultimately implemented using aspects.\\
\\
However, there are differences concerning the specific individual steps, the level of automation as well as the relating quality of the approaches. Approach 1 describes a systematology to derive aspect test cases from requirements. These have to be implemented manually. In contrast, approach 2 assumes the test objectives to be provided and generates corresponding test aspects automatically. Subsequently, the most significant difference is the tool support: approach 1 does not provide any tool support, but rather describes a systematology. Approach 2 fully automates the final step of approach 1 – from test objectives to test code – and assists with the set-up of the concrete test cases through the structure of the domain specific language.\\ 
\\
This is reflected in the evaluation of the respective approach as well. Both draw the conclusion that aspects are well suited for testing non-functional requirements. Approach 1 criticises the lack of tool support, approach 2 counteracts this exact problem. In Summary, the two approaches are quite similar. However, approach 1 is less extensive concerning the automation of the technique and should be regarded as a rather fundamental research. Approach 2 expands on the ideas of approach 1 and enhances their quality by partially automating the process. 

\newpage
\newgeometry{margin=1cm}
\begin{landscape}

\begin{table}
\caption{\textbf{Synthesis Matrix.}}
\begin{longtable}{|p{2cm}|>{\raggedright}p{4cm}|>{}p{8cm}|>{}p{8cm}|}
\hline
Question 
& Name 
& Approach 1: \textbf{Testing Non-Functional Requirements with Aspects} 
& Approach 2: \textbf{ACRE}\\ \hline
%\hline %%%%%%%%%%%%%%%%%%%%%%%%%%%%%%%%%%%%%%%%%%%%%%%%%%%%%%%%
\multirow{3}{*}{3 \rotatebox[origin=r]{90}{\textbf{Description}}} 
& a) artefacts and relationship between artefacts 
& system requirements in natural language (provided), NFRs (derived), testing objectives (derived), test aspects (derived)
& test objectives or bug report (provided), DSL-statements within the source code (derived), test aspects (generated)\\ 
\cline{2-4}
& b) preconditions/ input 
&  system requirements 
& test objectives or bug report\\ 
\cline{2-4}
& c) steps, results, informations 
&  \underline{Step 1}: Provided a set of system requirements, NFRs (which might be crosscutting) are derived. One NFR is derived from one or more system requirements, a system requirement can comprise multiple NFRs. \underline{Step 2}: The NFRs are categorised and corresponding testing objectives are derived. One testing objective is derived from one ore more NFRs. \underline{Step 3}: Test aspects can be formulated based on the testing objectives and NFR categories. One test aspect can comprise multiple testing objectives.
 & \underline{Step 1}: Provided a set of testing objectives or a bug report, the DSL statement describing the test aspect can be derived. The type of the bug or the category of the underlying NFR of the testing objective determines the type of the test aspect that must be chosen in this step. \underline{Step 2}: The test aspect is generated automatically based on the DSL description.\\
\hline %%%%%%%%%%%%%%%%%%%%%%%%%%%%%%%%%%%%%%%%%%%%%%%%%%%%%%%%
\multirow{3}{*}{4 \rotatebox[origin=r]{90}{\textbf{Benefits}}} 
& a) supported usage scenarios 
& Easy and non-invasive testing throughout the software lifecycle: during initial development for debugging or validating NFRs, during system testing and for maintaining tasks. Separation of concerns by modularizing crosscutting NFRs. Tracing of NFRs and corresponding tests 
& Easy and non-invasive testing throughout the software lifecycle: during initial development for debugging or validating NFRs, during system testing and for maintaining tasks. Separation of concerns by modularizing crosscutting NFRs. Tracing of NFRs and corresponding tests\\ 
\cline{2-4}
& b) supported stakeholders 
& Developers, testing and maintenance personnel
& Developers, testing and maintenance personnel\\ 
\cline{2-4}
& c) corresponding SWEBOK-Knowledge Areas
 & Software Requirements (NFRs, Requirements Tracing), Software Testing (System Test), Software Maintenance 
& Software Requirements (NFRs, Requirements Tracing), Software Testing (System Test), Software Maintenance\\ 
\hline %%%%%%%%%%%%%%%%%%%%%%%%%%%%%%%%%%%%%%%%%%%%%%%%%%%%%%%%
\multirow{2}{*}{5 \rotatebox[origin=r]{90}{\textbf{Tools}}} 
& a) tool support 
& none (except the aspects themselves)
&ACRE (Automated Aspect Creator), DSL\\ 
\cline{2-4}
& b) level of automation 
&  none 
& Test cases and testing objectives must be derived manually from the requirements. The DSL facilitates the test case design. The test aspect code is generated automatically from the DSL statements.\\ 
\cline{2-4}
\hline %%%%%%%%%%%%%%%%%%%%%%%%%%%%%%%%%%%%%%%%%%%%%%%%%%%%%%%%
\multirow{2}{*}{6 \rotatebox[origin=r]{90}{\textbf{Quality}}} 
& a) evaluation
& Case study with requirements of an existing industrial embedded system. Comparison of test coverage with and without test aspects.
& An empirical study aiming to find errors in the mathematical optimisation software NOMAD was conducted. The approach was contrasted with other common techniques and tools.\\ 
\cline{2-4}
& b) evaluation results 
& Test aspects proved to be an easy and non-invasive technique for testing NFRs. Seperation of (test) concerns by modularising NFRs.  BUT: lack of tool support, complicated build process. 
& Automatically generated test aspects proved to be suitable for finding memory leaks, interference bugs and testing invariants. The approach is easy to use because developers do not need to know aspect-oriented programming itself, only the DSL syntax and semantics. BUT: derivation of test cases from requirements and DSL statement input still manual\\ 
\cline{2-4}
\hline %%%%%%%%%%%%%%%%%%%%%%%%%%%%%%%%%%%%%%%%%%%%%%%%%%%%%%%%
\end{longtable}
\label{syn}
\end{table}

\end{landscape}
\restoregeometry

\newpage
\section{Conclusion} \label{feierabend}

In summary, this report presents two articles describing possible approaches to harness aspect-oriented techniques for testing non-functional requirements. A literature search based on the first article was carried out to obtain an overview of the current state of research and find relevant literature for this report. The selected article builds upon the ideas of the given article and presents a tool for (partially) automating the creation of test aspects. The two approaches have been compared and illustrated using a literature synthesis with a set of synthesis questions and a synthesis matrix as well as an application to an example.\\
\\ 
The results of both research articles and the synthesis and example, conducted for the report, indicate that aspects are suitable for testing non-functional requirements in a systematic manner due to multiple reasons: Firstly, the basic functionality of aspects carries an inherent systematology and lends itself for testing. An aspect comprises a set of statements that can be executed every time before, after or around a function call, which can easily be utilised to check pre- and postconditions or to measure resource consumption and runtimes. In addition to that, they are non-invasive, meaning the tested source code does not need to be modified and the test aspect can easily be separated from the main code, reused or removed. Furthermore, test aspects are especially suitable for testing non-functional requirements because they modularise the system level concerns, non-functional requirements pose, thus maintaining separation of concerns. System-wide functionality, that is spread out through the entire system and its source code can be tested using one aspect. This also improves traceability between the non-functional requirements and the corresponding tests.\\
\\ 
Metsä et al. point out that the lacking tool support and the need for developers to have a firm understanding of aspect-oriented programming pose possible challenges. Duclos et al. solve these problems (partially) by introducing ACRE, a tool that parses DSL statements to generate test aspects automatically.  However, the tool only supports four aspect types and is limited to C++ applications. The lack of subsequent research articles (after 2013) indicates that systematic testing using aspects is not in focus of current research anymore. Nevertheless, existing approaches and tools, as presented in this report, should be further refined to make use of the aforementioned advantages of aspect-oriented testing, especially for non-functional requirements. Because aspect-oriented programming lost popularity in recent years, applying aspect-oriented testing techniques in practice is difficult because only few developers and testers are able to implement them. However, using a domain specific language as an intermediate layer of abstraction, as described by Duclos et al, tackles this problem and enables relevant stakeholders to use test aspects as small, re-usable and none-invasive test modules throughout the software lifecycle at the system test level. 

\chapter{Conclusion} \label{chap:conclusion}

This chapter presents first of all the most important insights from each of the individual chapters. For this, the conclusions from each topic are summarized. Then, the improved knowledge areas from the SWEBOK are listed and their use over the chapters is compared. In the end, a final conclusion over all the topics is presented.

\textbf{Chapter \ref{sec:topic_2}} presents different approaches to create acceptance tests that can be automatically executed with the tool FitNesse. Due to the few results in the literature search, it seems that this topic using the specific tool FitNesse is not popular in the research. The first presented approach \cite{el-attar} is aimed at larger projects and uses lots of artefacts in the process. The second presented approach \cite{longo} is aimed to smaller projects and is adjusted for the use of US-UIDs, a type of artefact that was invented by some of the authors of the article. Whilst the first approach should be executed by a business analyst, the second approach is designed to be executed by the customer and the developers combined.  Both presented approaches are highly dependent on the experience and skill of the person executing the approach. Most of the steps are done manually and require human judgement. Therefore, the approaches are only recommended if the required artefacts are already part of the engineering process and a person with experience with these artefacts takes part in the engineering process.

\textbf{Chapter \ref{sec:topic_3}} showcases approaches to automatically derive test scenarios from means of the specification area using transition systems. This improves traceability between the specification and implementation and allows for an easy validation of requirements. The test cases can be derived by traversing the transition system. The first approach \cite{ClementineNebut2006} focuses on system level test generation whilst the second presented approach \cite{NajlaRaza2007} focuses on use case level test generation. Both presented approaches do not generate a sufficient amount of robustness tests for fault detection and do not sufficiently cover data variations. However, the presented approaches are still recommended for the automatic generation of functional test scenarios.

\textbf{Chapter \ref{sec:topic_4}} presents approaches to test real-time requirements. The literature search showed that this topic is not popular in the research. One of the presented approaches \cite{Siegl2010} involves manually executed, intermediate steps. Therefore, this approach cannot be recommended in the presented form. The other approach \cite{Guan2015} executes all intermediate steps automatically and can be recommended for usage.

\textbf{Chapter \ref{sec:topic_5}} focuses on testing with a classification tree. This technique offers the possibility to reduce the number of test cases. The first presented approach \cite{Conrad} describes how requirements can be transformed into a classification tree for the input parameters of a system model. Another presented article \cite{Belli} shows that classification trees are, despite existing, more recent methods, still relevant.

\textbf{Chapter \ref{sec:topic_6}} is missing.
\newpage
In \textbf{Chapter \ref{sec:topic_7}} the topic \textit{testing with system models} is discussed. The literature search showed that the articles relevant to this topic are mostly published between 2000-2009. Both presented approaches use requirements traceability in their model-based test generation processes and provide a tool support to generate tests automatically. The first approach \cite{Paper1} provides test generation tool Qtronic, which is not existing anymore. Therefore, the second approach \cite{Paper2} could be recommended for the automatic test generation with system models, because it provides LEIRIOS Test Designer tool (currently named Smartesting), which is still available. However, both approaches lack the detailed explanation on how the test cases are generated via the provided test generation tools.

\textbf{Chapter \ref{sec:topic_8}} presents approaches to automatically generate test cases for functional and non-functional requirements from User Requirements Notation. The literature search showed that this topic has not been researched enough, as this test generation process requires an intermediate step. The first presented approach \cite{ArnoldCorriveauShi2010} addresses both functional and non-functional requirements, however, the provided Validation Framework is not available and without it, this approach cannot be used. Although the second presented approach \cite{BoucherMussbacher2017} generates test cases only for functional requirements, it is still recommendable as the required frameworks, jUCMNav and JUnit, are still available and the whole process can be done quickly without problems. 

\textbf{Chapter \ref{sec:topic_9}} focuses on testing non-functional requirements with risk analysis. Both presented approaches deal with non-functional requirements and mention risk analysis only as a side note. Although the first presented approach \cite{ZouPavlovski2008} describes how non-functional requirements can be defined and controlled, it does not specify a way to test them except for simulating the operating condition. The second presented approach \cite{Lagerstedt2014} leaves it to software architects to define non-functional requirements and uses only architectural non-functional requirements that exist as code conventions and other guidelines. The conclusion is that non-functional requirements with higher risks need to be paid more attention to by giving the tests higher priority.

\textbf{Chapter \ref{sec:topic_10}} presents approaches to harness aspect-oriented techniques for testing non-functional requirements. While the first presented approach \cite{Metsa} points out the lacking tool support and the need for developers to have a firm understanding of aspect-oriented programming pose possible challenges, the second presented approach \cite{Duclos} solves these problems (partially) by introducing ACRE. ACRE is a tool that parses DSL statements to generate test aspects automatically, which only supports four aspect types and is limited to C++ applications. The literature search showed that systematic testing using aspects is not in focus of current research anymore due to its implementation difficulty. Nevertheless, existing approaches and tools should be further refined to make use of the advantages of aspect-oriented testing, especially for non-functional requirements. 

The improved knowledge areas of the SWEBOK according to the synthesis matrices of the individual chapters are the following:\\
Knowledge areas from the field of \textit{Software Testing} were improved by every topic. Improvements in the field of \textit{Software Requirements} are part of every topic except the topic presented in chapter \ref{sec:topic_5}. The topics presented in chapter \ref{sec:topic_5}, \ref{sec:topic_8}, \ref{sec:topic_9} and \ref{sec:topic_10} specifically improved a knowledge area from the field of \textit{Software Maintenance}. The field of \textit{Software Engineering Models and Methods} is improved by the topics presented in chapter \ref{sec:topic_3}, \ref{sec:topic_4}, \ref{sec:topic_7} and \ref{sec:topic_8}. Only the approaches from the topic of chapter \ref{sec:topic_2} and one of the approaches from the topic of chapter \ref{sec:topic_5} are mentioned to help with \textit{Software Construction}.

Furthermore, while the topics presented in chapter \ref{sec:topic_7} and \ref{sec:topic_10} are not popular in current research, the topics presented in chapter \ref{sec:topic_2}, \ref{sec:topic_4} and \ref{sec:topic_8} are generally not popular in research. However, still two different approaches could be found and are described for each topic of this work. Most of the topics of this work turned out to have potential to be automated at least partly. According to the synthesis matrices of the individual chapters other than the topics presented in chapter \ref{sec:topic_5} and \ref{sec:topic_9}, every topic included at least one approach that uses some automation steps for the creation of tests. Still, most of the work has to be done manually. Only the topic presented in chapter \ref{sec:topic_3} includes an approach that is mostly automated.

In Summary, most of the presented approaches are recommended under appropriate circumstances. These circumstances include having persons with the needed skills as in chapter \ref{sec:topic_2} or having access to specific tools as in chapter \ref{sec:topic_7} and \ref{sec:topic_8}. It should be mentioned that even though these approaches seem useful, often certain steps are not explained in detail. In particular, this is the case for the test generation steps such as in chapter \ref{sec:topic_3}, \ref{sec:topic_7}, and \ref{sec:topic_9}. 


%% Glossary
\glsaddall
\printglossary[title={Glossary},nonumberlist]

%% Bibliography
\bibliographystyle{plainnat}
%\bibliography{literature.bib}%Bibliography file name

\begin{thebibliography}{9}

\bibitem{KunChen2005} Chen, K, Zhang, W., Zhao, H.: An approach to constructing feature models based on requirements clustering.
In: 13th IEEE International Conference on Requirements Engineering (RE05). pp. 31-40. (2005)

\bibitem{RichtigesZitierenTUDresden} Institut für Geographie   
Lehrstuhl für Allgemeine Wirtschafts- und Sozialgeographie: An Hinweise zum wissenschaftlichen Arbeiten.
\url{http://www.geogr.uni-jena.de/fileadmin/Geoinformatik/Lehre/backup_05_2007/pdf-dokumente/Skript_WissArbeiten.pdf}

\bibitem{SWEBOK} PLATZHALTER

% https://ieeexplore.ieee.org/document/4578345/references#references
\bibitem{ZouPavlovski2008} J. Zou and C. J. Pavlovski, "Control Cases during the Software Development Life-Cycle," 2008 IEEE Congress on Services - Part I, Honolulu, HI, 2008, pp. 337-344, doi: 10.1109/SERVICES-1.2008.46.

% https://ieeexplore.ieee.org/abstract/document/6908675
\bibitem{Lagerstedt2014} R. Lagerstedt, "Using automated tests for communicating and verifying non-functional requirements," 2014 IEEE 1st International Workshop on Requirements Engineering and Testing (RET), Karlskrona, 2014, pp. 26-28, doi: 10.1109/RET.2014.6908675.

% https://ieeexplore.ieee.org/document/1438375
\bibitem{Andre_Search_1} A. Gregoriades and A. Sutcliffe, "Scenario-based assessment of nonfunctional requirements," in IEEE Transactions on Software Engineering, vol. 31, no. 5, pp. 392-409, May 2005, doi: 10.1109/TSE.2005.59.

% https://ieeexplore.ieee.org/document/5954452
\bibitem{Andre_Search_2} Z. A. Barmi, A. H. Ebrahimi and R. Feldt, "Alignment of Requirements Specification and Testing: A Systematic Mapping Study," 2011 IEEE Fourth International Conference on Software Testing, Verification and Validation Workshops, Berlin, 2011, pp. 476-485, doi: 10.1109/ICSTW.2011.58.

\bibitem{el-attar} El-Attar, M., Miller, J. Developing comprehensive acceptance tests from use cases and robustness diagrams. Requirements Eng 15, 285–306 (2010).

\bibitem{longo} Longo, D., Vilain, P., Pereira da Silva, L., Mello, R.: A web framework for test automation: user scenarios through user interaction diagrams. In Proceedings of the 18th International Conference on Information Integration and Web-based Applications and Services (iiWAS '16). Association for Computing Machinery, New York, NY, USA, 458–467 (2016). 

\bibitem{longo2} Longo, D. H., and Vilain, P.: Creating User Scenarios
through User Interaction Diagrams by Non-Technical
Customers. In 27th International Conference on Software
Engineering and Knowledge Engineering. 330-335 (2015).

\bibitem{acm} ACM Digital Library. \url{https://dl.acm.org/} 

\bibitem{fitnesse} FitNesse. \url{http://docs.fitnesse.org/FrontPage}

\bibitem{ieee} IEEE Xplore. \url{https://ieeexplore.ieee.org/Xplore/home.jsp}

\bibitem{springer} Springer Link. \url{https://link.springer.com/}

\bibitem{elsevier} Science Direct. \url{https://www.sciencedirect.com/}

\bibitem{Fit-tables} FitNesse: Writing Fit Tables. \url{http://fitnesse.org/FitNesse.UserGuide.WritingAcceptanceTests.FitFramework.WritingFitTables}

% ------ BEGIN REFERENCES HAUSBERGER ------ 

\bibitem{ClementineNebut2006} Nebut, C., Fleurey, F., Le Traon, Y., Jézéquel, J.-M.: Automatic Test Generation: A Use Case Driven Approach.
In: IEEE Transactions on Software Engineering (Volume: 32, Issue: 3, March 2006).
https://doi.org/10.1109/TSE.2006.22.

\bibitem{NajlaRaza2007} Raza, N., Nadeem, A., Iqbal, M. Z. Z.: An Automated Approach to System Testing based on Scenarios and Operations Contracts.
In: Seventh International Conference on Quality Software (QSIC 2007).
https://doi.org/10.1109/QSIC.2007.4385504.

\bibitem{FelixHausberger2020} Hausberger, F.: Research planning, research and mid-term presentation, \url{https://github.com/fidsusj/SWE-Seminar}, last accessed 2021/01/10.

\bibitem{MonalisaSarma2007} Sarma, M., Mall, R.: System Testing using UML Models.
In: 16th IEEE Asian Test Symposium (ATS 2007).
https://doi.org/10.1109/ATS.2007.102.

\bibitem{RosziatiIbrahim2007} Ibrahim, R., Saringat, M. Z., Ibrahim, N., Ismail, N.: An Automatic Tool for Generating Test Cases from the System's Requirements.
In: 7th IEEE International Conference on Computer and Information Technology (CIT 2007).
https://doi.org/10.1109/CIT.2007.116.

\bibitem{LizheChen2010} Chen, L., Li, Q.: Automated test case generation from use case: A model based approach.
In: 2010 3rd International Conference on Computer Science and Information Technology.
https://doi.org/10.1109/ICCSIT.2010.5563772.

\bibitem{XiaojingZhang2015} Zhang, X., Tanno, H.: Requirements document based test scenario generation for web application scenario testing.
In: 2015 IEEE Eighth International Conference on Software Testing, Verification and Validation Workshops (ICSTW).
https://doi.org/10.1109/ICSTW.2015.7107465.

\bibitem{LipingLi2008} Li, L., Miao, H.: An Approach to Modeling and Testing Web Applications Based on Use Cases.
In: 2008 International Symposium on Information Science and Engineering.
https://doi.org/10.1109/ISISE.2008.265.

\bibitem{YGKim1999} Kim, Y. G., Hong, H. S., Cho, S. M., Bae, D. H., Cha, S. D.: Test cases generation from UML state diagrams.
In: IEEE Proceedings - Software, Vol. 146, No. 4, pp. 187-192, Aug. 1999.
https://doi.org/10.1049/ip-sen:19990602.

\bibitem{ClementineNebut2003} Nebut, C., Fleurey, F., Le Traon, Y., Jézéquel, J.-M.: Requirements by Contracts allow Automated System Testing.
In: Proceedings of the 14th International Symposium on Software Reliability Engineering (ISSRE November 2003).
https://dl.acm.org/doi/10.5555/951952.952350.

% ------ END REFERENCES HAUSBERGER ------ 

% ------ BEGIN REFERENCES EDINGER ------ 

\bibitem{C++} AspectC++ Development Team: About the Project.
\url{https://www.aspectc.org/} last accessed 2020/12/28

\bibitem{Dijkstra} Dijkstra, E.: A Discipline of Programming.
In: Prentice-Hall Series in Automatic Computation Englewood Cliffs, New Jersey. (1976)

\bibitem{Duclos} Duclos, E., Le Digabel, S., Guéhéneuc, Y.,  Adams, B.: "ACRE: An Automated Aspect Creator for Testing C++ Applications“.
In:  CSMR 2013 Genova.  pp. 121-130. (2013). \url{10.1109/CSMR.2013.22}

\bibitem{J} Eclipse Foundation: AspectJ.
\url{https://www.eclipse.org/aspectj/}  last accessed 2020/12/28

\bibitem{Metsa} Metsä, J., Katara, M., Mikkonen, T.: "Testing Non-Functional Requirements with Aspects: An Industrial Case Study“.
In:  QSIC 2007 Portland. pp. 5-14. (2007). \url{10.1109/QSIC.2007.4385475}

\bibitem{Parnas} Parnas, D.: On the criteria to be used in decomposing systems into modules.
In: Communications of the ACM. (1972)

% ------ END REFERENCES EDINGER ------ 


% ------ BEGIN REFERENCES KOCH ------
\bibitem{Siegl2010} Siegl, S., Hielscher K.-S., German R.: "Model Based Requirements Analysis and Testing of Automotive Systems with Timed Usage Models". In: IEEE International Conference on Requirements Engineering. pp. 345-350. (2010)

\bibitem{Guan2015} Guan, J., Offutt, J.: "A Model-Based Testing Technique for Component-Based Real-Time Embedded Systems". In: IEEE Eighth International Conference on Software Testing. (2015)

\bibitem{imdb} IMDB. \url{https://www.imdb.com/}

% ------ END REFERENCES KOCH ------




% ------ BEGIN REFERENCES YAGMUR ARSLAN -TOPIC 7 ------
\bibitem{Paper1} F. Abbors, D. Truscan and J. Lilius, “Tracing Requirements in a Model-Based Testing Approach," 2009 First International Conference on Advances in System Testing and Validation Lifecycle, Porto, 2009, pp. 123-128, doi: 10.1109/VALID.2009.15.

\bibitem{matera} F. Abbors, A. Bäcklund and D. Truscan, “MATERA - An Integrated Framework for Model-Based Testing," 2010 17th IEEE International Conference and Workshops on Engineering of Computer Based Systems, Oxford, 2010, pp. 321-328, doi: 10.1109/ECBS.2010.46.

\bibitem{SMvsTM} Q. A. Malik et al., “Model-Based Testing Using System vs. Test Models - What Is the Difference?," 2010 17th IEEE International Conference and Workshops on Engineering of Computer Based Systems, Oxford, 2010, pp. 291-299, doi: 10.1109/ECBS.2010.41.

\bibitem{LRGuidelines} Chair of Software Engineering of University Heidelberg. Guidelines for literature research.
\url{https://confluence-se.ifi.uni-heidelberg.de/x/XQAyDg}

\bibitem{Relevant2} F. Abbors, A. Bäcklund and D. Truscan, “MATERA - An Integrated Framework for Model-Based Testing," 2010 17th IEEE International Conference and Workshops on Engineering of Computer Based Systems, Oxford, 2010, pp. 321-328, doi: 10.1109/ECBS.2010.46.

\bibitem{Relevant3} F. Bouquet, C. Grandpierre, B. Legeard, F. Peureux, N. Vacelet, and M. Utting. 2007. A subset of precise UML for model-based testing. In Proceedings of the 3rd international workshop on Advances in model-based testing (A-MOST '07). Association for Computing Machinery, New York, NY, USA, 95–104. DOI:https://doi.org/10.1145/1291535.1291545

\bibitem{Relevant4}F. Bouquet, E. Jaffuel, B. Legeard, F. Peureux, and M. Utting. 2005. Requirements traceability in automated test generation: application to smart card software validation. SIGSOFT Softw. Eng. Notes 30, 4 (July 2005), 1–7. DOI:https://doi.org/10.1145/1082983.1083282

\bibitem{Relevant5}Q. A. Malik et al., “Model-Based Testing Using System vs. Test Models - What Is the Difference?," 2010 17th IEEE International Conference and Workshops on Engineering of Computer Based Systems, Oxford, 2010, pp. 291-299, doi: 10.1109/ECBS.2010.41.

\bibitem{Paper2} E. Bernard and B. Legeard, “Requirements Traceability in the Model-Based Testing Process,"  in Software Engineering, ser. Lecture Notes in Informatics, vol. 106. Bttinger, Stefan and Theuvsen, Ludwig and Rank, Susanne and Morgenstern, Marlies, 2007, pp. 45–54.

\bibitem{MovieManager}Chair of Software Engineering of University Heidelberg. Movie Manager Example. 
\url{https://confluence-se.ifi.uni-heidelberg.de/x/IgC6Dg} 

\bibitem{netreqdia}t2informatik GmbH. Requirement Diagram.
\url{https://t2informatik.de/en/smartpedia/requirement-diagram/} 

\bibitem{traqml}Q. A. Malik, D. Truscan and J. Lilius, “Using UML Models and Formal Verification in Model-Based Testing," 2010 17th IEEE International Conference and Workshops on Engineering of Computer Based Systems, Oxford, 2010, pp. 50-56, doi: 10.1109/ECBS.2010.13.

\bibitem{LTG}Bernard, Eddy \& Bouquet, Fabrice \& Charbonnier, Amandine \& Legeard, Bruno \& Peureux, Fabien \& Utting, Mark \& Torreborre, Eric. (2006). Model-based testing from UML models. 94. 223-230. 

% ------ END REFERENCES YAGMUR ARSLAN -TOPIC 7 ------

% ------ BEGIN REFERENCES Ahmet Efe -TOPIC 8 ------

\bibitem{ArnoldCorriveauShi2010} Arnold, D., Corriveau, J.-P., and Shi, W., Scenario-Based Validation: Beyond the User Requirements Notation, 21st Australian Software Engineering Conf. (ASWEC 2010), IEEE CS, pp. 75–84, 2010. DOI:10.1109/ASWEC.2010.29.

\bibitem{BoucherMussbacher2017} Boucher, M., Mussbacher, G.: Transforming Workflow Models into Automated End-to-End Acceptance Test Cases, Proc. - 2017 IEEE/ACM 9th International Workshop on Modelling in Software Engineering. MiSE 2017, pp. 68–74, 2017. DOI:10.1109/MiSE.2017.5.

% ------ END REFERENCES Ahmet Efe -TOPIC 8 ------


\end{thebibliography}

\listoffigures

\listoftables

\end{document}          
